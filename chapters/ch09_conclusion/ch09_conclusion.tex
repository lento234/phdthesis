\chapter{Conclusion and outlook}
\label{ch:conclusions}


\section{Main results and conclusions}

%\begin{itemize}
%	\item Vegetation provides benefit to the urban climate, but most importantly, not always.
%\end{itemize}

\subsection{Contributions to the research field}

%\begin{itemize}
%	\item Implemented a vegetation model in OpenFOAM.
%\end{itemize}

\section{Outlook and further research}

%\begin{itemize}
%	\item Improved radiation model: Approaches such as discrete ordinate method (DOM) should be employed to explicitly model the absorption, scattering and transmission of short-wave radiation through vegetation. Vegetation is modeled as a participating media.
%	\item Turbulence modelling: Other RANS approaches, LES simulation
%	\item Refractive index matching experiments
%	\item Field measurements
%	\item Validation study with BUBBLE
%	\item Data assimilation for calibrating the numerical model.
%	\item Thermal capacity in leaf energy balance
%	\item Heterogenous soil properties
%	\item Groundwater modelling
%	\item Rain interception
%	\item Pollutant dispersion impact of vegetation
%	\item Seasonal variation: changing leaf area density, drag, latent heat flux...
%	\item Nested simulation, driven by large scale model (ex. COSMO)
%	\item Plant competition.
%	\item Biotic influences on plant physiology. 
%	\item Influence of soil salinity (osmotic potential) on plant physiology. Especially in cities, salt deposition during winter seasons can have adverse effects on the soil salinity, potentially impacting the plant health. 
%	\item Cohesion-tension theory to model water and sap transport in xylem and phloem respectively.
%	\item Influence of global climate change. Climate change results in increased average air temperature, CO$_2$ concentration and a change in rainfall events. The net results would be a change from the present predictions. 
%	\item And finally, machine learning.
%	\item Inverse uncertainty quantification: Estimate bias of mathematical model with experiments. Perform parameter calibration for the unknown parameters of the model.  
%	\item Investigating plant physiology and climatic impact of plants employing $C_4$ carbon fixation instead of the typical $C_3$ carbon fixation. The $C_4$ plants are known to employ a more efficient photosynthetic process with higher water use efficiency, outperforming at conditions with high water stress and higher temperature. Therefore, such plant are ideal for surviving in urban climate. However, studies on urban thermal comfort impact of such plants are lacking.
%\end{itemize}