\chapter{Conclusion and outlook}
\label{ch:conclusions}

This chapter provides the conclusions obtained from various difference chapters in this thesis. The thesis employed both experimental and numerical approaches to provide an assessment on the impact of vegetation on the urban microclimate. The experimental campaigns consisted of wind tunnels measurements in an atmospheric boundary layer (ABL) wind tunnel where the impact of vegetation is determined by firt studying the influence of vegetation on the airflow and secondly the influence of vegetation on the overall climate. The first measurement campaign focused on isothermal flow conditions where model and small natural trees were used. The impact on the airflow was determined using a load cell to measure the drag force and a particle image velocimetry (PIV) to measure the wake flow. The second measurement campaign focused on the hygrothermal flow parameters such as air temperature, relative humidity, solar radiation, leaf temperature along with the wake velocity statistics. Towards this, PIV, infrared thermography and hygrothermal sensor analysis were employed. Furthermore, to link the plant porosity or the eaf area density distribution to the flow modification, x-ray tomography was employed to reconstruct the topology foliage. In addition to experiments, impact of vegetation on urban microclimate was numerical predicted using an integrated, multi-domain coupled, vegetation model. It is developed as a computation fluid dynamics (CFD) model in OpenFOAM. The developed numerical approach simultaneously resolved turbulence modification, radiation balance, heat and mass fluxes, and the sensitivity to soil moisture. The influence of the water availability on the transpiration rate is modeled using a soil-plant-atmosphere continuum (SPAC) model.  Thus, the model was able to quantify the cooling potential of vegetation along with its impact on thermal comfort for a pedestrian.

The goal of the present thesis was to accurately assess the impact of vegetation at a microclimate scale, enabling a more accurate development of urban heat island (UHI) mitigation strategies for improving human health and thermal comfort in cities.

\section{Main results and conclusions}

%\begin{itemize}
%	\item Vegetation provides benefit to the urban climate, but most importantly, not always.
%\end{itemize}

\subsection{Contributions to the research field}

This thesis presents various novel applications of experimental and numerical techniques towards understanding the impact of vegetation. The contribution of this thesis consists of experimental understanding on the impact of vegetation on the airflow and the microclimate. Furthermore, a novel vegetation integrated urban microclimate model is presented to understand the influence of vegetation on the atmosphere coupled with the soil properties. The scientific contributions can be summarized as:

%\begin{itemize}
%	\item Implemented a vegetation model in OpenFOAM.

\begin{itemize}
	\item Development of an integrated, multi-domain coupled vegetation model in \texttt{OpenFOAM}. The \textit{air} domain solver and the \textit{solid} domain solver is linked together with the soil-plant-atmosphere continuum (SPAC) modeling approach to model the water transport due to vegetation. An advanced stomatal model is implemented that responds to water availability in addition to the atmospheric evaporative demand (AED).
	
	\item Development of a modified urban radiation model that takes in account of short-wave and long-wave radiation integration of vegetation with the urban surfaces in \texttt{OpenFOAM}
	
	\item Development of a thermal comfort index analysis technique that quantifies the impact of vegetation in an urban setting.

	\item A quantitative understanding of the sheltering provided by model and small natural trees and how it differs from a mature tree. The influence of vegetation on the airflow is studied in an atmospheric boundary layer (ABL) wind tunnel using small model and natural trees. In the study, the turbulent airflow behind the trees is studied using particle image velocimetry (PIV) measurement technique and is linked to the drag force measurements using a load cell. 
	
	\item A high-resolution dataset of \textit{Buxus sempervirens} obtained through multiple non-intrusive imaging techniques. The dataset consists of wake velocity statistics obtained through particle image velocimetry (PIV), plant foliage temperature using infrared thermography, diurnal variation of net plant transpiration rate using a mass balance, and a high-resolution the plant microstructure metrics using  X-ray tomography. The high-resolution dataset can be employed for comparing with numerical models as demonstrated in \cref{ch:wtcfdcomparison}

	\item A novel application of X-ray tomography to quantify the plant microstructure metrics plant porosity and the leaf area density distribution. 

	\item A rigorous parametric study on the impact of vegetation on the transpirative cooling potential. An understanding on how much the environmental factors (i.e., wind speed, air temperature, relative humidity, and solar radiation intensity) and tree properties (i.e., leaf size, stomatal resistance, and leaf area density) contribute to the plant cooling performance and the pedestrian comfort is provided. 
	
	\item A study on the impact of vegetation of urban microclimate of street-canyon and the influence of water availability.
	
	\item Application of integrated vegetation model in a realistic setup of Muensterhof (Zurich, Switzerland) with a more realistic tree geometry. 
	
\end{itemize}


\section{Outlook and further research}

\begin{itemize}
	\item \textit{Improved radiation model}: Approaches such as discrete ordinate method (DOM) should be employed to explicitly model the absorption, scattering and transmission of short-wave radiation through vegetation. Vegetation is modeled as a participating media.
	
%	\item \textit{Turbulence modeling}: Improving the turbulence modeling approach with other RANS approaches or even opting for large-eddy simulation (LES) strategy. The vegetation model can be expanded to investigate additional eddy-viscosity models such as $k-\omega$ or $k-\omega$ SST, Reynolds stress models (RSM), and even Large-eddy simulation (LES) . A more complex, computationally expensive turbulence modeling approach has been shown to improve the prediction accuracy} \citep{Hiraoka2011,Yue2008,Lopes2013}.

	\item \textit{Turbulence inside plant}: Refractive-index-matching (RIM) tunnel experiments can be employed to study the flow inside transparent models. Therefore, the turbulence inside the foliage could be experimentally measured using transparent vegetation models.

	\item \textit{Validation study with BUBBLE or Field measurements}
	
	\item \textit{Varying drag coefficient}: In this thesis, we demonstrated that drag coefficient is a function of wind speed. In future, this could be incorporated to improve the prediction of sheltering provided by vegetation. 
	
	\item \textit{Groundwater modeling}: In this thesis, the ground water level was not explicitly modeled. In future, the water table can be driving boundary condition for assessing water availability.

	\item \textit{Rain interception}: The interception of rain due to foliage is an important aspect in wind driven rain. The present modeling approach can be easily extended to describe the influence of vegetation on wind driven rain.

	\item \textit{Pollutant dispersion}: One of the fundamental aspects of vegetation is the sheltering provided by vegetation. The implication of this on the pollutant dispersion characterstics is an important aspect of future research. 
	
	\item \textit{Seasonal variation}: It is known that the  leaf area density varies monthly for certain species. This the shading and the sheltering provided by vegetation is dependent on the month. Furthermore, deciduous species do not provide transpiration after autumn, directly affecting the transpirative cooling potential.
	
	\item \textit{Nested simulation}: A nested simulation approach can be employed to drive the urban microclimate model with a large scale model (e.g. COSMO). The advantage is to provide a more realistic boundary condition for assessing the impact of vegetation.

	\item \textit{Plant competition}: Plant competition such as for sun-light or soil moisture could effect the plant physiology and thereby the impact on the environment. Such implication to modify our understanding on the impact of vegetation on urban microclimate. For example a configuration of grass only or grass with tree foliage above could result in different responses from the grass on the climate.
	
	\item \textit{Biotic influences}: The hydraulic redistribution from deep-rooted plants was observed in this study. The influence of this on other species can be investigated in future.

	\item \textit{Soil salinity}: The soil salinity (i.e., osmotic potential) plays an important role on plant physiology. Especially in cities, salt deposition during winter seasons can have adverse effects on the soil salinity, potentially impacting the plant health. 
	
	\item \textit{Sap flow}: The Cohesion-tension theory to model water and sap transport in xylem and phloem respectively. This approach can be an alternative to the presently used simplifed bulk xylem transport model. The benefits include differentiating the xylem layers and providing a higher level of complexity in describing the water transport inside the plant.
	
	\item \textit{Climate change}: The influence of climate change results in increased average air temperature, CO$_2$ concentration and a change in rainfall events. The net results would be a change from the present predictions and is vital aspect for future research.
	
	\item \textit{Inverse uncertainty quantification}: Such approaches can be used to estimate bias of mathematical model based on experimental data. The method can enable parameter calibration for the unknown parameters of the model.  
	
	\item \textit{$C_4$ plant species}:	Investigating plant physiology and climatic impact of plants employing $C_4$ carbon fixation instead of the typical $C_3$ carbon fixation. The $C_4$ plants are known to employ a more efficient photosynthetic process with higher water use efficiency, outperforming at conditions with high water stress and higher temperature. Therefore, such plant are ideal for surviving in urban climate. However, studies on urban thermal comfort impact of such plants are lacking.
	
	\item \textit{Machine learning}: Machine learning (ML) is a popular field of research now. Towards this growing demand and applicability, ML can be introduced at various facets of the present research. For example, the plant water transport model can be encoded into an ML model possible reducing the run-time computation code. 
\end{itemize}