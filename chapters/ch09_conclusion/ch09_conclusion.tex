\chapter{Conclusion and outlook}
\label{ch:conclusions}

This chapter provides the conclusions obtained from the chapters in this thesis. The goal of the present thesis was to accurately assess the impact of vegetation at a microclimate scale, enabling a more accurate assessment of urban heat island (UHI) mitigation strategies for improving human health and thermal comfort in cities.

\section{Main results and conclusions}

The thesis employed both experimental and numerical approaches to provide an assessment of the impact of vegetation on the urban microclimate. The experimental campaigns consisted of wind tunnels measurements in an atmospheric boundary layer (ABL) wind tunnel where the impact of vegetation is determined by first studying the influence of vegetation on the airflow and secondly the influence of vegetation on the overall climate. The first measurement campaign focused on isothermal flow conditions where model and small natural trees were used. Smaller plants had to be used to ensure minimal blockage effect inside the tunnel. The impact on the airflow was determined using a load cell to measure the drag force and a particle image velocimetry (PIV) to measure the wake flow. The second measurement campaign focused on the hygrothermal flow parameters such as air temperature, relative humidity, solar radiation, leaf temperature along with the wake velocity statistics. Towards this, stereoscopic particle image velocimetry (SPIV), infrared thermography and hygrothermal sensor analysis were employed. To link the plant porosity to the wake flow configuration, X-ray tomography was employed to reconstruct the topology of the foliage. Furthermore, a need for an integrated approach that couples the conditions of the atmosphere with soil properties was seen to determine the impact of water availability on the transpirative cooling potential. Therefore, in this thesis, a coupled model was developed to assess the hygrothermal impact of vegetation on the urban microclimate. The soil-plant-atmosphere continuum (SPAC) model was used to link the plant transpiration with the soil moisture. The developed numerical approach simultaneously resolved turbulence modification, radiation balance, heat and mass fluxes, and the sensitivity to soil moisture. Finally, the net effect of all these factors on the pedestrian thermal comfort was investigated using the mean radiant temperature and the universal thermal climate index (UTCI).

In this thesis, it was demonstrated that the behavior of natural trees can be approximated by model trees only if they have a similar aerodynamic porosity and if they are capable of reconfiguring at high wind speeds. The PIV measurements showed that low porosity model trees have a similar wake structure as of a natural cypress tree. In the study, it was evident that the foliage where branches and leaves deform due to airflow in both drag measurements and wake velocity statistics. It was seen that due to reconfiguration, the drag coefficient decays at high wind speeds. Young natural trees and model trees with artificial leaves showed reconfiguration resulting in a linear drag-force-wind speed relationship. The Vogel exponent was used to identify the strength of the reconfiguration. The study showed that the aerodynamic porosity and the drag coefficient are vital parameters that should be matched to those of the natural trees of interest. When studying the influence of reconfiguration of the plant on the flow field, the Vogel exponent of the model tree is a recommended parameter to compare model and natural trees.

It was observed that there is no apparent link between the spatial distribution of the plant porosity and the spatial variation in turbulent kinetic energy (TKE). The TKE intensity was seen to be governed by the net plant porosity and the outer geometry of the plant foliage that generates the shear-layer and they determine how the flow interacts with the upstream boundary layer profile. However, a parametric study of the plant drag forces showed that the wake TKE intensity is negatively correlated with the total plant porosity. At high plant porosity (i.e., very low plant drag), vegetation has less impact on the TKE. It was also shown that the aerodynamic and optical porosities typically used in the wind tunnel studies do not reflect the true porosity distribution of the plant. 

The diurnal variation in the leaf temperature and the net plant transpiration rate enabled us to quantify the diurnal hysteresis resulting from the stomatal response lag. Experimentally, the plant was seen to comprise of four stages \textit{no-cooling} (i.e., the stage when stomata have not responded to the increase in solar radiation),\textit{ high-cooling} (i.e., when stomatal response tries to compensate for increased leaf temperature), \textit{equilibrium} (i.e., when stomatal response and leaf temperature equilibrates) and \textit{decaying-cooling} stage (i.e., when the transpiration rate starts to weaken). 

The transpirative cooling effect of a single row of trees is highest at lower wind speed when $U<1$ m\,s$^{-1}$. An increase in vegetation height was also seen to be beneficial as the top of the trees with higher leaf temperatures is further away from the pedestrian level. This ensures that the transpirative cooling effect is higher at the pedestrian level. To improve the UHI, it is best achieved by maximizing the sensible heat extraction. It was observed that increasing the vegetation density and tree height strongly correlated with an increase in sensible heat extraction and a reduction in air temperature. Thus, increasing vegetation in cities is seen to be the best strategy to improve global UHI. Moreover, cities should use a combination of tall wide-canopy trees, that can provide shading to urban surfaces, and pedestrian-level trees, that can provide transpirative cooling near the ground. Such a combination can maximize the cooling through shading and transpiration. It was seen that plant shading contributes more to the improvement of the pedestrian thermal comfort than the plant transpiration. The transpiration only has a direct influence on the plant vicinity air temperature and negligible effects on the thermal comfort measured through the Universal Thermal Climate Index (UTCI). In contrast, the shading provided by vegetation has a large influence as indicated by the drop in mean radiant temperature in the shadow. A large drop in the mean radiant temperature substantially improves the measured thermal comfort. 

It was determined that the nocturnal radiation trapping due to the presence of vegetation is an important aspect that should be considered. At night, due to obstruction of long-wave radiation emission from urban surfaces to the sky by vegetation, the mean radiant temperature and equally UTCI is higher during night time. So, at night, vegetation can dampen the cooling of cities and thereby negatively affect the UHI.

The plant is seen to redistribute the water from deep ground to the upper layer of the soil through hydraulic redistribution. This can have an important contribution to the water availability of the shallow-rooted plant species such as grass or small shrubs, showing the importance of a mixed vegetation setup in urban areas. 

Finally, an important aspect that also plays an important role is water availability. With an increasing number of days without irrigation, the soil was seen to lose the available water for the plant. We observed that due to this, the transpiration rate decays over time, resulting in a reduced transpirative cooling as determined through an increase in sensible heat flux. Moreover, a prolonged period without irrigation showed that water stress increases exponentially every day.  


\subsection{Contributions to the research field}

This thesis presents various novel applications of experimental and numerical techniques towards understanding the impact of vegetation on the urban microclimate. The contribution of this thesis consists of experimental understanding of the impact of vegetation on the airflow and the microclimate. Furthermore, a novel vegetation integrated urban microclimate model is presented to understand the influence of vegetation on the atmosphere, coupled with the soil properties. The scientific contributions can be summarized as as follows:

\begin{itemize}
	\item Development of an integrated, multi-domain coupled vegetation model in OpenFOAM. The \textit{air} domain solver and the \textit{solid} domain solver is linked together with the soil-plant-atmosphere continuum (SPAC) modeling approach to model the water transport due to vegetation. An advanced stomatal model is implemented that responds to water availability in addition to the atmospheric evaporative demand (AED).
	
	\item Development a novel radiation modeling approach for describing short-wave and long-wave interaction of vegetation in an urban environment.
	
	\item Development of a thermal comfort index analysis technique that quantifies the impact of vegetation in an urban setting.

	\item A quantitative understanding of the sheltering provided by model and small natural trees and how it differs from a mature tree. The influence of vegetation on the airflow is studied in an atmospheric boundary layer (ABL) wind tunnel using small model and natural trees. In the study, the turbulent airflow behind the trees is studied using particle image velocimetry (PIV) measurement technique and is linked to the drag force measurements using a load cell. 
	
	\item A high-resolution data\-set of \textit{Buxus sempervirens} obtained through multiple non-intrusive imaging techniques. The data\-set consists of wake velocity statistics obtained through particle image velocimetry (PIV), plant foliage temperature using infrared thermography, diurnal variation of net plant transpiration rate using a mass balance, and a high-resolution the plant microstructure metrics using  X-ray tomography. The high-resolution data\-set can be employed for comparing with numerical models as demonstrated in \cref{ch:wtcfdcomparison}

	\item A novel application of X-ray tomography to quantify the plant microstructure metrics plant porosity and the leaf area density distribution. 

	\item A rigorous parametric study on the impact of vegetation on the transpirative cooling potential. An understanding is provided on how much environmental factors (i.e., wind speed, air temperature, relative humidity, and solar radiation intensity) and tree properties (i.e., leaf size, stomatal resistance, and leaf area density) can affect to the plant cooling performance and the pedestrian comfort.
	 
	\item A study on the impact of vegetation on urban microclimate of street-canyon and the influence of water availability.
	
	\item Application of integrated vegetation model in a realistic setup of Muensterhof (Zurich, Switzerland) with a more realistic tree geometry. 
	
\end{itemize}


\section{Outlook and further research}

\begin{itemize}
	\item \textit{Improved radiation model}: Approaches such as discrete ordinate method (DOM) should be employed to explicitly model the absorption, scattering, and transmission of short-wave radiation through vegetation \citep{Gastellu-Etchegorry1996,Bailey2014,Sinoquet2001}.
	
	\item \textit{Turbulence modeling}: The vegetation turbulence model can be expanded to investigate additional eddy-viscosity models such as $k-\omega$ or $k-\omega$ SST, Reynolds stress models (RSM), and even Large-eddy simulation (LES). A more complex, computationally expensive turbulence modeling approach has been shown to improve the prediction accuracy \citep{Hiraoka2011,Yue2008,Lopes2013}.

	\item \textit{Turbulence inside plant}: Refractive-index-matching (RIM) tunnel experiments can be employed to study the flow inside transparent models. The turbulence inside the foliage could be experimentally measured using transparent vegetation models in such tunnels \citep{Weitzman2014, Harshani2017,Bai2014a,Bai2012}. So, RIM experiments of model trees can be employed to validate and improve turbulence modeling of vegetation.
	
	\item \textit{Validation of the full-model with field measurements}: One of the missing element in the present study is the validation of the full numerical model. Field measurements (such as the BUBBLE measurement campaign \citep{Rotach2005}), can provide a means of validating the full numerical model. The validation requires the conditions in the soil region, and the measurements of the plant responses such as stomatal conductance, root conductance and xylem conductance. 
	
	\item \textit{Varying drag coefficient}: In this thesis, we demonstrated that the drag coefficient is a function of wind speed. In the future, this could be incorporated to improve the prediction of sheltering provided by vegetation. 
	
	\item \textit{Groundwater modeling}: The groundwater level was not modeled in the thesis. In the future, the water table can be a driving boundary condition for assessing water availability providing a more accurate estimation of the soil moisture. 
	
	\item \textit{Precipitation and irrigation}: Soil moisture also depends on environmental conditions such as rainfall or irrigation. Therefore, modeling the change in soil moisture due to them can provide a more accurate estimation of the spatiotemporal variability in water availability for plants. Furthermore, rainfall dynamics such as rainwater drainage and rainwater runoff can additionally improve the prediction of the distribution of soil moisture.
	
	\item \textit{Rain interception}: The interception of rain due to foliage is an important aspect of wind-driven rain. The present model can be extended to model the influence of vegetation on wind-driven rain.

	\item \textit{Pollutant dispersion}: One of the fundamental aspects of vegetation is the wind sheltering provided by vegetation. The implication of this on the pollutant dispersion characteristics is an important aspect of future research. 
	
	\item \textit{Seasonal variation}: It is known that the leaf area density varies monthly for certain species. So, the shading and the sheltering provided by vegetation is dependent on the month. As deciduous species do not provide transpiration after autumn, the impact on transpirative cooling potential should be studied.
	
	\item \textit{Nested simulation}: A nested simulation approach can be employed to drive the urban microclimate model with a large mescoscale climate model (e.g., COSMO). The advantage is to provide more realistic boundary conditions for assessing the impact of vegetation.

%	\item \textit{Plant competition}: Plant competition such as for sunlight or soil moisture could affect the plant physiology and thereby the impact on the environment. For example, a configuration of grass only versus grass and tree configuration could result in different responses on the climate.
	
	\item \textit{Biotic influences}: The hydraulic redistribution from deep-rooted plants was observed in this study. The influence of this on other species can be investigated in the future such as the physiology of surface-bounded grass layer.

	\item \textit{Soil salinity}: The soil salinity (i.e., osmotic potential) plays an important role in plant physiology. Especially in cities, salt deposition during winter seasons can have adverse effects on the soil salinity, potentially impacting the plant health. 
	
	\item \textit{Sap flow}: The Cohesion-tension theory can be used to model water and sap transport in xylem and phloem, respectively,  and the resulting water potential inside the plant. This approach can be an alternative to the presently used simplified bulk xylem transport model with a bulk value for the leaf, root, and xylem water potentials. The benefits include resolving the hydraulic architecture of the plant, the various xylem layers, and its influence on water transport. Plant water storage and release, known to affect the water transport dynamics, can be also be modeled.
		
	\item \textit{Climate change}: The influence of climate change results in increased average air temperature, CO$_2$ concentration and a change in rainfall events. These environmental changes could have implications on the predicted transpirative cooling potential of vegetation.
	
	\item \textit{Inverse uncertainty quantification}: Such approaches can be used to estimate bias of a mathematical model based on experimental data. The method can enable parameter calibration for the unknown parameters of the model.  
	
	\item \textit{$C_4$ plant species}:	Investigating plant physiology and climatic impact of plants employing $C_4$ carbon fixation instead of the typical $C_3$ carbon fixation. The $C_4$ plants are known to employ a more efficient photosynthetic process with higher water use efficiency, outperforming at conditions with high water stress and higher temperature. Therefore, such a plant is ideal for surviving in urban climate. However, studies on urban thermal comfort impact of such plants are lacking.
	
	\item \textit{Different climates}: The present thesis focused on the oceanic climate of Zurich (i.e., \textit{Cfb}). The transpiration cooling potential of vegetation was seen to be directly dependent on the atmospheric evaporative demand (AED) and, so, studies on different climates such tropical climates (i.e., \textit{Af, Am, Aw/As}; cities such as Singapore) or Mediterranean climates (i.e., \textit{Csa, Csb, Csc}; cities such as Rome), can provide a more universal assessment of vegetation as UHI mitigation strategy.
	
	\item \textit{City-scale assessment}: In the present thesis, the microclimate assessment has been confined to a city-square or smaller. In the future, the evaluation of the influence of vegetation can be expanded to the scale of a full city to determine the overall effect of vegetation in a city. Although, some aspects of the present modeling approach might need to be simplified to ensure computational tractability of the model.
	
	\item \textit{Further case studies}: The influence of vegetation can be better understood through additional cases studies, investigating different tree-building configurations, various plant size, shape, species, density, and impact of building materials properties.
	
	\item \textit{Application of machine learning}: Machine learning (ML) is a popular field of research now. Towards this growing demand and applicability, ML can be introduced at various facets of the present research. For example, the plant water transport model can be encoded into an ML model potentially providing a substantial improvement in the run-time computational performance. 
	
\end{itemize}
