% document options

%------------------------------------------------------------
% Formatting packages
\usepackage{blindtext}
\usepackage{scrlayer-scrpage}
%\usepackage[raggedright]{titlesec}
\usepackage[titletoc]{appendix}
\usepackage{rotating}
\usepackage{setspace}
%\usepackage{flipbook}
%------------------------------------------------------------

%------------------------------------------------------------
% character encoding and languages
%\usepackage[russian, american, ngerman]{babel}
\usepackage[american, ngerman]{babel}
\usepackage[T2A, T1]{fontenc}
\usepackage[utf8]{inputenc}
%\input glyphtounicode
%\pdfgentounicode=1
%------------------------------------------------------------

%------------------------------------------------------------
% math packages
%\usepackage{bm}
\usepackage{amsthm}
\usepackage{amsmath}
\usepackage{amsfonts}
\usepackage{amssymb}
\usepackage{mathtools}
%\usepackage{math}
\usepackage{xfrac}
\usepackage{siunitx} % use SI unit

%------------------------------------------------------------
\DeclareSymbolFont{sfletters}{OML}{cmbrm}{b}{it}

\DeclareMathSymbol{\salpha}{\mathord}{sfletters}{"0B}
\DeclareMathSymbol{\sbeta}{\mathord}{sfletters}{"0C}
\DeclareMathSymbol{\sgamma}{\mathord}{sfletters}{"0D}
\DeclareMathSymbol{\sdelta}{\mathord}{sfletters}{"0E}
\DeclareMathSymbol{\sepsilon}{\mathord}{sfletters}{"0F}
\DeclareMathSymbol{\szeta}{\mathord}{sfletters}{"10}
\DeclareMathSymbol{\seta}{\mathord}{sfletters}{"11}
\DeclareMathSymbol{\stheta}{\mathord}{sfletters}{"12}
\DeclareMathSymbol{\siota}{\mathord}{sfletters}{"13}
\DeclareMathSymbol{\skappa}{\mathord}{sfletters}{"14}
\DeclareMathSymbol{\slambda}{\mathord}{sfletters}{"15}
\DeclareMathSymbol{\smu}{\mathord}{sfletters}{"16}
\DeclareMathSymbol{\snu}{\mathord}{sfletters}{"17}
\DeclareMathSymbol{\sxi}{\mathord}{sfletters}{"18}
\DeclareMathSymbol{\spi}{\mathord}{sfletters}{"19}
\DeclareMathSymbol{\srho}{\mathord}{sfletters}{"1A}
\DeclareMathSymbol{\ssigma}{\mathord}{sfletters}{"1B}
\DeclareMathSymbol{\stau}{\mathord}{sfletters}{"1C}
\DeclareMathSymbol{\supsilon}{\mathord}{sfletters}{"1D}
\DeclareMathSymbol{\sphi}{\mathord}{sfletters}{"1E}
\DeclareMathSymbol{\schi}{\mathord}{sfletters}{"1F}
\DeclareMathSymbol{\spsi}{\mathord}{sfletters}{"20}
\DeclareMathSymbol{\somega}{\mathord}{sfletters}{"21}
\DeclareMathSymbol{\svarepsilon}{\mathord}{sfletters}{"22}
\DeclareMathSymbol{\svartheta}{\mathord}{sfletters}{"23}
\DeclareMathSymbol{\svarpi}{\mathord}{sfletters}{"24}
\DeclareMathSymbol{\svarrho}{\mathord}{sfletters}{"25}
\DeclareMathSymbol{\svarsigma}{\mathord}{sfletters}{"26}
\DeclareMathSymbol{\svarphi}{\mathord}{sfletters}{"27}
%------------------------------------------------------------

%------------------------------------------------------------
% thesis style
\usepackage[%
%    drafting,
%    minionpro,
%    minionprospacing,
    dottedtoc,
    floatperchapter,
    parts,
	eulerchapternumbers,
]{classicthesis}
%\usepackage{MnSymbol}
%\usepackage[authordate]{biblatex-chicago}

\titleformat{\chapter}[display]%
{\relax}{\vspace*{0\baselineskip}\makebox[\linewidth][r]{\color{halfgray}\chapterNumber\thechapter}}{15pt}%
{\large\raggedright\spacedallcaps}[\normalsize\vspace*{.8\baselineskip}\titlerule]%


%------------------------------------------------------------

%------------------------------------------------------------
% define page size and margins for the final print
\usepackage{geometry}
% for e-version
%\geometry{paperwidth=17.00cm,paperheight=24.589cm,
% for print, with large margin
\geometry{paperwidth=18.91cm,paperheight=24.589cm,
bindingoffset=0.89cm, %print version
%bindingoffset=0cm, % e-version
inner=2cm, % inside margin
outer=2.5cm, %for print version
%%outer=2cm, %for e-version
vmargin=2.5cm, % top and bottom margins
%marginparsep=0.6cm,
%marginparwidth=3cm,
}
%------------------------------------------------------------

%------------------------------------------------------------
% Counters
\setcounter{secnumdepth}{3}
\setcounter{tocdepth}{3}
%------------------------------------------------------------

%------------------------------------------------------------
% Figures, Tables, captions
\usepackage{tabularx}
\usepackage{ltablex} 
\usepackage{longtable}
\newcommand{\tableheadline}[1]{\multicolumn{1}{c}{\spacedlowsmallcaps{#1}}}
%------------------------------------------------------------


%------------------------------------------------------------
% algorithms
\usepackage[chapter]{algorithm}
\usepackage{algorithmic}
%\usepackage{listing} % 
%\usepackage{minted} % for source codes
%\usepackage{ifthen}
\usepackage{enumitem}

%------------------------------------------------------------


%------------------------------------------------------------
% notation
\usepackage{style/notation}
%------------------------------------------------------------


%------------------------------------------------------------
% fixes for marginpar and latex
\usepackage{mparhack}
\usepackage{fixltx2e} 
\usepackage{tocstyle}

%------------------------------------------------------------


%------------------------------------------------------------
% indexing
\usepackage{makeidx}
\makeindex
%------------------------------------------------------------

%------------------------------------------------------------
% graphics, figures, captions
\usepackage{graphicx}
\usepackage{caption}
\usepackage{subcaption}
\usepackage{tikz} % flowchart
%------------------------------------------------------------

%------------------------------------------------------------
% tikz setup

% Flowchart definition
\usetikzlibrary{calc,trees,positioning,arrows,chains,shapes.geometric,%
	decorations.pathreplacing,decorations.pathmorphing,shapes,%
	matrix,shapes.symbols}

\tikzset{
	>=stealth',
	punktchain/.style={
		rectangle, 
		%rounded corners, 
		fill=black!10,
		%draw=black, % very thick,
		text width=15em, 
		minimum height=3em, 
		text centered, 
		on chain},
	line/.style={draw, thick, <-},
	element/.style={
		tape,
		top color=white,
		bottom color=blue!50!black!60!,
		minimum width=8em,
		draw=blue!40!black!90, very thick,
		text width=10em, 
		minimum height=3.5em, 
		text centered, 
		on chain},
	every join/.style={->, thick,shorten >=1pt},
	decoration={brace},
	tuborg/.style={decorate},
	tubnode/.style={midway, right=2pt},
}

\tikzset{
	>=stealth',
	punktchainN/.style={
		rectangle, 
		%rounded corners, 
		fill=black!5,
		%draw=black, % very thick,
		text width=15em, 
		minimum height=3em, 
		text centered, 
		on chain},
	line/.style={draw, thick, <-},
	element/.style={
		tape,
		top color=white,
		bottom color=blue!50!black!60!,
		minimum width=8em,
		draw=blue!40!black!90, very thick,
		text width=10em, 
		minimum height=3.5em, 
		text centered, 
		on chain},
	every join/.style={->, thick,shorten >=1pt},
	decoration={brace},
	tuborg/.style={decorate},
	tubnode/.style={midway, right=2pt},
}

\tikzset{
	>=stealth',
	punktchainO/.style={
		rectangle, 
		dashed,
		%rounded corners, 
		%fill=black!10,
		draw=black!80, % very thick,
		text=black!80,
		text width=15em, 
		minimum height=3em, 
		text centered, 
		on chain},
	%line/.style={draw, thick, <-},
	element/.style={
		tape,
		top color=white,
		bottom color=blue!50!black!60!,
		minimum width=8em,
		draw=blue!40!black!90, very thick,
		text width=10em, 
		minimum height=3.5em, 
		text centered, 
		on chain},
	every join/.style={->, thick,shorten >=1pt},
	decoration={brace},
	tuborg/.style={decorate},
	tubnode/.style={midway, right=2pt},
}

\tikzset{
	>=stealth',
	punktchainS/.style={
		rectangle, 
		draw=red,
		%rounded corners, 
		fill=red!30,
		%draw=black, % very thick,
		text width=15em, 
		minimum height=3em, 
		text centered, 
		on chain},
	%line/.style={draw, thick, <-},
	element/.style={
		tape,
		top color=white,
		bottom color=blue!50!black!60!,
		minimum width=8em,
		draw=blue!40!black!90, very thick,
		text width=10em, 
		minimum height=3.5em, 
		text centered, 
		on chain},
	every join/.style={->, thick,shorten >=1pt},
	decoration={brace},
	tuborg/.style={decorate},
	tubnode/.style={midway, right=2pt},
}

% Tikz stypes
%\tikzstyle{every node}=[draw=black,thick,anchor=west]
%\tikzstyle{optional}=[dashed,fill=gray!20]
%\tikzstyle{selected}=[draw=red,fill=red!30]

% Tikz stypes
\tikzstyle{every node}=[draw=black,thick,anchor=west]
\tikzstyle{optional}=[dashed,fill=gray!20]
\tikzstyle{selected}=[draw=red,fill=red!30]

\tikzstyle{line}=[draw,-latex']

\tikzstyle{class}=[draw,fill=blue!20]
\tikzstyle{module}=[draw,fill=orange!50!red!50!white]
\tikzstyle{script}=[draw,fill=yellow!20]
\newcommand\mybox[2][]{\tikz[overlay]\node[fill=blue!20,inner sep=2pt, anchor=text, rectangle,#1] {#2};\phantom{#2}}
%------------------------------------------------------------


%------------------------------------------------------------
% table
\usepackage{style/slashbox}
\usepackage{rotating}
\usepackage{multirow}
%\usepackage{booktabs}
%%\usepackage{longtable}
\usepackage{ctable} % better tables
%\usepackage{booktabs} %

%------------------------------------------------------------


%------------------------------------------------------------
% new colors

\definecolor{flautuiorange}{HTML}{f39c12}

\definecolor{flatuired}{HTML}{e74c3c}
\definecolor{flatuidarkred}{HTML}{c0392b}

\definecolor{flatuigreen}{HTML}{2ecc71}
\definecolor{flatuidarkgreen}{HTML}{27ae60}

\definecolor{flatuiblue}{HTML}{3498db}
\definecolor{flatuidarkblue}{HTML}{2980b9}

\definecolor{flatuiturquoise}{HTML}{1abc9c}
\definecolor{flatuigreensea}{HTML}{16a085}
\definecolor{flatuipurple}{HTML}{8e44ad}
\definecolor{flatuiorange}{HTML}{e67e22}

\definecolor{flatuidark}{HTML}{2c3e50}
\definecolor{flatuigrey}{HTML}{95a5a6}
\definecolor{flatuilight}{HTML}{ecf0f1}

%------------------------------------------------------------



%------------------------------------------------------------
% hyperref
\usepackage{hyperref}
\hypersetup{%
    %draft,	% = no hyperlinking at all (useful in b/w printouts)
    colorlinks=true, linktocpage=true, pdfstartpage=1, pdfstartview=FitV,%
    % uncomment the following line if you want to have black links (e.g., for printing)
    %colorlinks=false, linktocpage=false, pdfborder={0 0 0}, pdfstartpage=1, pdfstartview=FitV,% 
    breaklinks=true, pdfpagemode=UseNone, pageanchor=true, pdfpagemode=UseOutlines,%
    plainpages=false, bookmarksnumbered, bookmarksopen=true, bookmarksopenlevel=1,%
    hypertexnames=true, pdfhighlight=/O,%hyperfootnotes=true,%nesting=true,%frenchlinks,%
    urlcolor=webbrown, linkcolor=RoyalBlue, citecolor=webgreen, %pagecolor=RoyalBlue,%
%    urlcolor=flatuidarkred, linkcolor=flatuidarkblue, citecolor=flatuidarkgreen, %pagecolor=RoyalBlue,%
%    urlcolor=Black, linkcolor=Black, citecolor=Black, %pagecolor=Black,%
    pdftitle={\myTitle},
    pdfauthor={\myName},
    pdfsubject={},%
    pdfkeywords={},%
}
%------------------------------------------------------------


%------------------------------------------------------------
% Custom commands
\renewcommand{\subsubsectionautorefname}{Subsubsection}   
\renewcommand{\subsectionautorefname}{Subsection}
\renewcommand{\sectionautorefname}{Section}
\renewcommand{\chapterautorefname}{Chapter}
\def\algorithmautorefname{Algorithm}
\renewcommand{\arraystretch}{1.2}
%------------------------------------------------------------


%------------------------------------------------------------
% bibliography

%If you are compiling your thesis on linux (or OSX), use the backend biber

% \usepackage[backend=biblatex,
\usepackage[backend=biber,
style=authoryear,
dashed=false,
doi=false,
isbn=false,
url=false,
natbib=false,
maxcitenames=2,
%sortcites=true,
sorting=nyt,
maxbibnames=99,
%firstinits=true,
%uniquename=false,
%useprefix=true,
]{biblatex}
%style=chicago-authordate,

%natbib=true,

\newcommand{\citep}[1]{\autocite{#1}}
\renewcommand{\cite}[1]{\textcite{#1}}

%\addbibresource{frontbackmatter/bibliography.bib}

%\addbibresource{/Users/lento/Documents/phd/literature/library.bib} % mac empa
%\addbibresource{/Users/Lento/Documents/phd/publications/library.bib} % personal 

\addbibresource{frontbackmatter/library_final.bib}

%\bibliography{../../publications/library.bib}
%\addbibresource{../../literature/library.bib}
%\bibliographystyle{spbasic}      % basic style, author-year citations
%------------------------------------------------------------


%------------------------------------------------------------
% theorems, definitions
%\newtheorem{theorem}{Theorem}[chapter]
%\newtheorem{lemma}{Lemma}[chapter]
%\providecommand*{\lemmaautorefname}{Lemma}
%\newtheorem{proposition}[theorem]{Proposition}
%\theoremstyle{definition}
%\newtheorem{definition}[theorem]{Definition}
%\newtheorem{example}[theorem]{Example}
%\theoremstyle{remark}
%\newtheorem*{remark}{Remark}

%\theoremstyle{definition}
\newtheoremstyle{mystyle}% name
{\topsep}% Space above
{\topsep}% Space below
{\itshape}% Body font
{}% Indent amount
{\bfseries}% Theorem head font
{:}%Punctuation after theorem head
{.5em}%Space after theorem head
{}% theorem head spec
\theoremstyle{mystyle}
\newtheorem*{assumption}{Assumption}
%\newtheorem*{assumption}{Assumption}[section]
%------------------------------------------------------------


%------------------------------------------------------------
% acronyms
\PassOptionsToPackage{printonlyused,smaller}{acronym}
\usepackage{style/acronym} % nice macros for handling all acronyms in the thesis
\renewcommand{\bflabel}[1]{{#1}\hfill} % fix the list of acronyms
%------------------------------------------------------------

% dummy counter for correct TOC etc
\newcounter{dummy}
%------------------------------------------------------------

%------------------------------------------------------------
% for cover page
\usepackage{pdfpages}
%------------------------------------------------------------

%------------------------------------------------------------
% Font design
% \usepackage{mathptmx}
%\usepackage{lmodern}
\usepackage{mathpazo}
%\usepackage[scaled]{helvet}
\usepackage{textcomp}
\usepackage{textgreek}
\usepackage{epigraph}
\usepackage{soul,color}
\usepackage[capitalize]{cleveref}
\usepackage{lettrine}
\renewcommand{\LettrineTextFont}{}
%------------------------------------------------------------

%------------------------------------------------------------	
% Custom shortcuts
\usepackage{style/shortcuts}
%------------------------------------------------------------

%------------------------------------------------------------
% Chapter thumbs (in the end)
\usepackage{style/chapterthumbs}
%------------------------------------------------------------

%\usepackage[bordercolor=white,backgroundcolor=gray!30,linecolor=black,colorinlistoftodos]{todonotes}
%\newcommand{\updated}[1]{\todo[color=yellow,inline]{#1}}


%\newcommand{\highlight}[1]{\hl{#1}}

\newcommand{\highlight}[1]{\textcolor{blue}{#1}}

\newcommand{\highlightpara}[1]{\textcolor{blue}{#1}}

%\newcommand{\highlightpara}[1]{\noindent\colorbox{yellow}{%
%	\parbox{\dimexpr\linewidth-2\fboxsep}% a box with line-breaks that's just wide enough
%	{#1} } }