
%\pdfbookmark[1]{Abstract}{Abstract}
%\refstepcounter{dummy}

\begingroup
\let\clearpage\relax
\let\cleardoublepage\relax
\let\cleardoublepage\relax
\refstepcounter{dummy}

\chapter{Abstract}

\thispagestyle{empty}

%This thesis proposes an integrated multiscale methodology to study wetting and drying in macroporous media exposed to environmental loads. The motivation behind this work arises from the fact that the durability of many porous materials exposed to the environment is significantly influenced by the residence time of water in them under typical environmental conditions. In this thesis, multiscale experimental investigations and numerical modeling of fluid transport are performed for a highly complex, macroporous material, namely porous asphalt (PA), with the aim to determine the most important environmental conditions and material characteristics that determine the residence time of water in PA. 
%
%At the smallest scale, neutron radiography experiments of imbibition, drainage and drying in PA, using a custom-built mini wind tunnel, are coupled with three-dimensional pore space characterization using X-ray microcomputed tomography to understand the physics of air and water transport in PA. The interaction of airflow with PA is further investigated using two techniques. At the smaller scale, computational fluid dynamics (CFD) simulations of airflow in a real PA geometry are performed to identify the influence of air entrainment on convective vapor removal from the internal pore space of PA. In addition, full-scale wind tunnel experiments are coupled with particle imaging velocimetry to characterize the air boundary layer above the air-PA interface and to identify the organized, turbulent flow structures in the boundary layer. 
%
%Since gravity-driven drainage and drying are found to be the most important factors determining the residence time of water in PA, pore network model (PNM) simulations of drainage in PA and subsequent drying of the residual liquid are performed. Gravity-driven drainage is simulated using a modified invasion percolation algorithm that includes the effect of gravity while drying is simulated using a cluster-based approach. An insight into the evolution of capillary pressure with saturation for macroporous media during gravity-driven drainage is obtained. It is also observed that a higher hydrophobicity in the pore network, simulated by increasing the number of pores that do not retain water after drainage, leads to shorter constant drying rate periods (CDRP) at the beginning of the drying simulations. To model the experimental observation of no CDRP at the beginning of drying in PA, a further improvement in the pore network is required in order to capture the high hydrophobicity and complex connectivity in PA as they are responsible for the low hydraulic connectivity and high vapor diffusion resistance during drying in PA. Finally, the PNM simulations are coupled to a heat-air-moisture continuum approach to show that the macroscopic drying behavior of a complex macroporous medium can be modeled with such a coupling approach, subject to more realistic pore networks. Such a continuum approach to drying is extremely computationally efficient and can be used to easily understand the influence of dynamic environmental loads on the drying process of a macroporous medium. 

Vegetation in cities is seen as an effective strategy to combat the growing UHI as it provided natural cooling. Vegetation offers natural cooling primarily by providing shading to urban structures and additionally by extracting heat from the surroundings during the photosynthesis driven transpiration process. However, the effectiveness of transpirative cooling is directly related to the water availability of the plants, and in extreme environmental conditions such as drought, the effectiveness of vegetation can be severely compromised. Furthermore, trees can obstruct ventilation which can have a negative impact on the pollution dispersion characteristics in cities. Therefore, the net impact of vegetation on pedestrian comfort and health in cities is dependent on various conditions and the most effective configuration for UHI mitigation is a non-trivial problem. Thus, an urban microclimate model that can model the airflow, radiation and the water cycle in an integrated approach is necessary for accurately assessing the impact of vegetation in a complex urban environment.

The thesis aims at establishing a more accurate and detailed prediction of the thermal influence of vegetation in an urban environment by simultaneously taking in account of its heat, mass and momentum exchanges and the influence of the water availability. The goal is to provide better guidelines for effective mitigation strategies with vegetation for any given urban, vegetation configuration and environmental conditions. The cooling potential of vegetation such as trees on the microclimate of a street-canyon is studied using a computational fluid dynamics (CFD) model in OpenFOAM. The flow field is numerically modeled by solving the Reynolds-averaged Navier-Stokes equations (RANS) with realizable $k-\varepsilon$ turbulence closure model. The vegetation model is integrated into the CFD solver as a porous medium, providing the necessary source terms for heat, mass and momentum fluxes, with additional closures for turbulence enhancement due to vegetation. A radiation model is developed to model the short-wave and long-wave radiative heat fluxes between the leaf surface and the surrounding. The radiation model enables to model the impact of the diurnal variation of solar intensity and direction, and the long-wave radiative fluxes between vegetation and nearby urban surfaces. Also, the water cycle driven by the transpiration process is explicitly modeled by coupling with an integrated soil heat and moisture dynamics model. A soil-plant-atmosphere continuum modeling approach is essential as the transpiration rate through the stomata is directly linked to the water availability at the roots of the plant. Therefore, the proposed method helps us understand the response of vegetation during extreme environmental conditions such as drought and provides a more accurate prediction towards the cooling performance of vegetation. The present study investigates the influence of transpirative and shaded cooling due to vegetation on pedestrian comfort inside a street canyon. The influence of various vegetation features such as size, shape, and density is studied to determine the optimal configuration for improving pedestrian comfort and health.

The thesis also employs wind tunnel experiments to provide a deeper understanding of the influence of an isolated tree on the flow. A comparative study of drag force and wake flow field of small model and natural trees shows the distinction between their responses and provide an insight into the challenges of representing trees in urban flow wind tunnel studies with model trees. Furthermore, the microclimate measurement of the small natural plant provide an understanding of the dynamic response of a plant and more a basis for the validation of the numerical model.

\vskip 5cm

\vfill

\pagebreak

\selectlanguage{ngerman}
\refstepcounter{dummy}
\chapter{Zusammenfassung}
\thispagestyle{empty}
%\foreignlanguage{ngerman}{
Vegetation in St\"adten wird als wirksame Strategie zur Bek\"ampfung des zunehmenden urbanen Hitzeinsel-Effekts (UHI) angesehen, da sie eine nat\"urliche K\"uhlung bietet. Die Vegetation erzeugt natürliche Kühlung in erster Linie durch Schattenbildung und zusätzlich durch Wärmeentnahme aus der Umgebung durch den Transpirationsprozess während der Photosynthese. Die Wirksamkeit der transpirativen Kühlung hängt jedoch direkt mit der Wasserverfügbarkeit der Pflanzen zusammen. Bei extremen Umweltbedingungen wie Dürre kann die Wirksamkeit der Vegetation stark beeinträchtigt werden. Andererseits können Bäume die Belüftung behindern, was sich negativ auf die Ausbreitungseigenschaften der Luftverschmutzung in Städten auswirken kann. Daher ist der Nettoeffekt der Vegetation auf Fußgängerkomfort und -gesundheit in Städten von verschiedenen Bedingungen abhängig und die effektivste Konfiguration für die UHI-Minderung ein nichttriviales Problem. Daher ist ein Mikroklimamodell für ein Stadtgebiet erforderlich, das den Luftstrom, die Strahlung und den Wasserkreislauf in einem integrierten Ansatz modellieren kann, um die Auswirkungen der Vegetation auf eine komplexe städtische Umgebung genau zu bewerten.

Das Ziel dieser Dissertation ist es, eine genauere und detailliertere Vorhersage des thermischen Einflusses der Vegetation in einer städtischen Umgebung zu ermöglichen, indem gleichzeitig der Austausch von Wärme, Masse und Impuls sowie der Einfluss der Wasserverfügbarkeit berücksichtigt werden. Das Ziel ist es, bessere Richtlinien für effektive Schutzstrategien des lokalen Klimas für jede gegebene Stadt, Vegetationskonfiguration und Umweltbedingungen bereitzustellen. Das Kühlungspotenzial von Vegetation wie Bäumen im Mikroklima eines Street-Canyons wird mit einem numerischen Strömungsmechanikmodell (CFD) in OpenFOAM untersucht. Das Strömungsfeld wird numerisch modelliert, indem die Reynolds-gemittelten Navier-Stokes-Gleichungen (RANS) mit einem ``realizable $ k- \varepsilon $'' Turbulenz modell gelöst. Das Vegetationsmodell ist als poröses Medium in den CFD-Solver integriert und liefert die notwendigen Quellterme für Wärme-, Massen- und Impulsflüsse. Ein Strahlungsmodell wird entwickelt, um die kurz- und langwelligen Strahlungswärmeflüsse zwischen der Blattoberfläche und der Umgebung zu modellieren. Das Strahlungsmodell ermöglicht die Modellierung des Einflusses der tageszeitlichen Variation der Sonnenintensität und -richtung sowie der langwelligen Strahlungsflüsse zwischen Vegetation und nahegelegenen Oberflächen. Der Wasserkreislauf, der durch den Transpirationsprozess angetrieben wird, wird auch explizit durch Kopplung mit einem integrierten Bodenwärme- und Feuchtedynamikmodell modelliert. Ein Modellierungsansatz für die Boden-Pflanzen-Atmosphäre-Kontinuität ist wesentlich, da die Transpirationsrate durch die Stomata direkt mit der Verfügbarkeit von Wasser an den Wurzeln der Pflanze zusammenhängt. Daher hilft das vorgeschlagene Verfahren, die Reaktion der Vegetation unter extremen Umweltbedingungen wie Dürre zu verstehen, und bietet eine genauere Vorhersage hinsichtlich der Kühlleistung der Vegetation. Die vorliegende Studie untersucht den Einfluss von transpirativer und schattierter Kühlung aufgrund von Vegetation auf den Fußgängerkomfort in einem Street-Canyon. Der Einfluss verschiedener Vegetationsmerkmale wie Größe, Form und Dichte wird untersucht, um die optimale Konfiguration zur Verbesserung des Komforts und der Gesundheit von Fußgängern zu bestimmen.

In dieser Arbeit werden auch Windkanalexperimente durchgeführt, um ein tieferes Verständnis des Einflusses eines isolierten Baums auf die Strömung zu vermitteln. Eine vergleichende Studie über das Widerstandsfeld zwischen Zugkraft und Nachlauf von kleinen Modell- und Naturbäumen zeigt den Unterschied ihrer Reaktionen und liefert einen Einblick in die Herausforderungen der Einbeziehung von Bäumen in ``Urban Flow''-Windkanalstudien mit Modellbäumen. Darüber hinaus liefert die Mikroklimamessung der kleinen natürlichen Pflanze ein Verständnis der dynamischen Reaktion einer Pflanze und dazu eine Grundlage für die Validierung des numerischen Modells.

\selectlanguage{american}

\vfill

\pagebreak

\endgroup

\vfill
