\begingroup
\let\clearpage\relax
\let\cleardoublepage\relax
\let\cleardoublepage\relax

\chapter{Abstract}

%This thesis proposes an integrated multiscale methodology to study wetting and drying in macroporous media exposed to environmental loads. The motivation behind this work arises from the fact that the durability of many porous materials exposed to the environment is significantly influenced by the residence time of water in them under typical environmental conditions. In this thesis, multiscale experimental investigations and numerical modeling of fluid transport are performed for a highly complex, macroporous material, namely porous asphalt (PA), with the aim to determine the most important environmental conditions and material characteristics that determine the residence time of water in PA. 
%
%At the smallest scale, neutron radiography experiments of imbibition, drainage and drying in PA, using a custom-built mini wind tunnel, are coupled with three-dimensional pore space characterization using X-ray microcomputed tomography to understand the physics of air and water transport in PA. The interaction of airflow with PA is further investigated using two techniques. At the smaller scale, computational fluid dynamics (CFD) simulations of airflow in a real PA geometry are performed to identify the influence of air entrainment on convective vapor removal from the internal pore space of PA. In addition, full-scale wind tunnel experiments are coupled with particle imaging velocimetry to characterize the air boundary layer above the air-PA interface and to identify the organized, turbulent flow structures in the boundary layer. 
%
%Since gravity-driven drainage and drying are found to be the most important factors determining the residence time of water in PA, pore network model (PNM) simulations of drainage in PA and subsequent drying of the residual liquid are performed. Gravity-driven drainage is simulated using a modified invasion percolation algorithm that includes the effect of gravity while drying is simulated using a cluster-based approach. An insight into the evolution of capillary pressure with saturation for macroporous media during gravity-driven drainage is obtained. It is also observed that a higher hydrophobicity in the pore network, simulated by increasing the number of pores that do not retain water after drainage, leads to shorter constant drying rate periods (CDRP) at the beginning of the drying simulations. To model the experimental observation of no CDRP at the beginning of drying in PA, a further improvement in the pore network is required in order to capture the high hydrophobicity and complex connectivity in PA as they are responsible for the low hydraulic connectivity and high vapor diffusion resistance during drying in PA. Finally, the PNM simulations are coupled to a heat-air-moisture continuum approach to show that the macroscopic drying behavior of a complex macroporous medium can be modeled with such a coupling approach, subject to more realistic pore networks. Such a continuum approach to drying is extremely computationally efficient and can be used to easily understand the influence of dynamic environmental loads on the drying process of a macroporous medium. 

Vegetation in cities is seen as an effective strategy to combat the growing UHI as it provided natural cooling. Vegetation offers natural cooling primarily by providing shading to urban structures and additionally by extracting heat from the surroundings during the photosynthesis driven transpiration process. However, the effectiveness of transpirative cooling is directly related to the water availability of the plants, and in extreme environmental conditions such as drought, the effectiveness of vegetation can be severely compromised. Furthermore, trees can obstruct ventilation which can have a negative impact on the pollution dispersion characteristics in cities. Therefore, the net impact of vegetation on pedestrian comfort and health in cities is dependent on various conditions and the most effective configuration for UHI mitigation is a non-trivial problem. Thus, an urban microclimate model that can model the airflow, radiation and the water cycle in an integrated approach is necessary for accurately assessing the impact of vegetation in a complex urban environment. 

The thesis aims at establishing a more accurate and detailed prediction of the thermal influence of vegetation in an urban environment by simultaneously taking in account of its heat, mass and momentum exchanges and the influence of the water availability. The goal is to provide better guidelines for effective mitigation strategies with vegetation for any given urban, vegetation configuration and environmental conditions. The cooling potential of vegetation such as trees on the microclimate of a street-canyon is studied using a computational fluid dynamics (CFD) model in OpenFOAM. The flow field is numerically modeled by solving the Reynolds-averaged Navier-Stokes equations (RANS) with realizable $k-\varepsilon$ turbulence closure model. The vegetation model is integrated into the CFD solver as a porous medium, providing the necessary source terms for heat, mass and momentum fluxes, with additional closures for turbulence enhancement due to vegetation. A radiation model is developed to model the short-wave and long-wave radiative heat fluxes between the leaf surface and the surrounding. The radiation model enables to model the impact of the diurnal variation of solar intensity and direction, and the long-wave radiative fluxes between vegetation and nearby urban surfaces. Also, the water cycle driven by the transpiration process is explicitly modeled by coupling with an integrated soil heat and moisture dynamics model. A soil-plant-atmosphere continuum modeling approach is essential as the transpiration rate through the stomata is directly linked to the water availability at the roots of the plant. Therefore, the proposed method helps us understand the response of vegetation during extreme environmental conditions such as drought and provides a more accurate prediction towards the cooling performance of vegetation. The present study investigates the influence of transpirative and shaded cooling due to vegetation on pedestrian comfort inside a street canyon. The influence of various vegetation features such as size, shape, and density is studied to determine the optimal configuration for improving pedestrian comfort and health. 

The thesis also employs wind tunnel experiments to provide a deeper understanding of the influence of an isolated tree on the flow. A comparative study of drag force and wake flow field of small model and natural trees shows the distinction between their responses and provide an insight into the challenges of representing trees in urban flow wind tunnel studies with model trees. Furthermore, the microclimate measurement of the small natural plant provide an understanding of the dynamic response of a plant and more a basis for the validation of the numerical model.

\vskip 5cm

\vfill

%\pagebreak
%
%\chapter{Zusammenfassung}

%Cette thèse propose une méthodologie multi-échelle intégrée pour étudier le mouillage et le séchage de matériaux macroporeux exposés à des charges environnementales. La motivation derrière ce travail provient du fait que la durée de vie d'un grand nombre de matériaux poreux exposés à l'environnement est fortement influencée par le temps de séjour de l'eau sous conditions ambiantes typiques. Dans cette thèse, des recherches expérimentales multi-échelles et des modélisations numériques du transport des fluides sont effectuées pour un matériau macroporeux très complexe, à savoir le revêtement bitumineux drainant (RBD), dans le but de déterminer les conditions environnementales et les caractéristiques du matériau qui déterminent le temps de séjour de l'eau dans le RBD.
%
%À la plus petite échelle, des expériences utilisant la radiographie à neutrons pour documenter l'imbibition, le drainage et le séchage d’éprouvettes de RBD placées dans une mini-soufflerie faite sur mesure, sont couplées à une caractérisation tridimensionnelle du système poreux obtenue par microtomographie à rayons X pour comprendre le transport de l’air et de l’eau dans le matériau. L'intéraction du flux d'air avec le RBD est en outre étudiée en utilisant deux techniques. À petite échelle, des simulations de flux d'air dans une géométrie réelle de RBD sont effectuées par dynamique computationnelle des fluides pour identifier l'influence de l'entraînement de l'air sur l'élimination par convection de la vapeur d’eau des pores de RBD. En outre, des expériences en soufflerie à grande échelle utilisent la vélocimétrie par imagerie de particules afin de caractériser la couche limite d'air au-dessus de l'interface air-RBD et d'identifier les structures d'écoulement turbulent organisé dans la couche limite.
%
%Étant donné que le drainage par gravité et le séchage se trouvent être les facteurs les plus importants déterminant le temps de séjour de l'eau dans le RBD, des simulations par modèle de réseau de pores (MRP) du drainage dans le RBD et du séchage subséquent du liquide résiduel sont effectués. Le drainage par gravité est simulé en utilisant un algorithme modifié d’invasion par percolation qui inclut l'effet de la gravité. Quant au séchage, il est simulé en utilisant une approche fondée sur la méthode des grappes. L'évolution de la pression capillaire avec le degré de saturation du matériau macroporeux pendant le drainage par gravité est bien obtenue. On observe également qu’un caractère hydrophobe plus élevé dans le réseau de pores, simulé en augmentant le nombre de pores qui ne retiennent pas l'eau après le drainage, mène à de plus courtes périodes constantes du taux de séchage (PCTS) au début de la simulation de séchage. Pour modéliser l’absence de PCTS au début du séchage, tel que observé expérimentalement, une amélioration du modèle part réseau de pores serait nécessaire afin de capturer la forte hydrophobie et la connectivité complexe du RBD, responsables de la basse connectivité hydraulique et de la résistance à la diffusion de vapeur élevée notées pendant le séchage du RBD. Enfin, la simulation par MRP couplée à un modèle en milieux continus du transport de la chaleur, de l’air et de l’eau modéliseraient adéquatement le comportement macroscopique du séchage d'un milieu macroporeux complexe, sous réserve de l’obtention de réseaux de pores plus réalistes. Une telle approche de modèle en milieux continus pour le séchage est extrêmement efficace informatiquement et peut être facilement utilisée pour comprendre l'influence des charges environnementales sur le processus de séchage d'un matériau macroporeux.


\vfill

\pagebreak

\endgroup

\vfill
