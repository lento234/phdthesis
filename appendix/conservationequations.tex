\chapter{Conservation equations of moist fluid flow}
\label{app:conservation}

This appendix consists of detailed derivation of conservation of \textit{mass}, \textit{momentum}, and \textit{energy} of moist fluid flow. Additional details are also provided in the thesis of \cite{Defraeye2011}. The following chapter derives the conservation equation in the conservative form. 

\section{Conservation principle}
\label{sec:Conservationprinciple}
Let us consider a fluid flow in Euclidean vector space ($\mathbb{R}^3$, $||\cdot||$) with coordinate vector $\mvec{x} = (x,y,z) \in \mathbb{R}^3$. The volume of fluid of interest (or control volume CV) is $\Omega \in \mathbb{R}^3$, bounded by a surface $\partial\Omega$. In the domain, any extensive property in $\Omega$ is defined as an integral of its intensive property:
\begin{equation}
\Phi \left( \mvec{x},t \right) \equiv \int_\Omega  \phi  \left( \mvec{x},t \right)\;dV
\end{equation}
where $\Phi$ is the extensive property of interest, and $\phi$ is the intensive property. For example, the extensive property mass $m$ (kg), is the integral of intensity property density $\rho$ (kg~m$^{-3}$) of the fluid. 

The Reynolds transport theorem is defined as:
\begin{quote}
	\centering
	Rate of increase of $\Phi$ in domain $\Omega$\\
	$=$\\
	Net rate of transfer of $\phi$ due to \textit{advection} at boundary $\partial \Omega$\\
	$+$\\
	Net rate of transfer of $\phi$ due to \textit{diffusion} at boundary $\partial \Omega$\\
$+$\\	
	Net rate of transfer of $\phi$ within domain $\Omega$ (i.e., source/sink)
\end{quote}

In simpler terms, the rate of increase of any extensive property $\Phi$ over time in the domain $\Omega$ is simply due to rate of transfer of the intensive property $\phi$ at the boundary of the domain $\Omega$ (i.e., $\partial \Omega$) due to \textit{convection} and the net rate of transfer of intensive property $\phi$ within the domain $\Omega$ (i.e., introduced into the domain as source of sinks). It must be note that a \textit{convection} is the sum of \textit{advection} (transport due to velocity) and \textit{diffusion} (transport due to molecular transfer). The rate of transfer of $\phi$ at the boundary is also commonly referred to as a \textit{flux} of $\phi$ at the boundary. A mathematical formulation of the Reynolds transport theorem is given as:
\begin{equation}
\frac{\mathrm{d}~}{\mathrm{d}t} \Phi (\mvec{x},t) = \frac{\mathrm{d}~}{\mathrm{d}t}\int_{\Omega} \phi\;\mathrm{d}V = -\int_{\partial\Omega} \phi~\mvec{u}\cdot\hat{\mvec{n}}\;\mathrm{d}A + \int_{\Omega} s\;\mathrm{d}V
\label{eq:reynoldstransportheromem}
\end{equation}
where $\mvec{u}$ is velocity, $\hat{\mvec{n}}$ is the normal vector (\textit{pointing outward}, hence the negative sign), and $s$ is the \textit{positive} source term in $\Omega$.

Using the divergence theorem,  surface integral can be converted into volume integral, i.e.:
\begin{equation}
\int_{\partial\Omega} \phi~\mvec{u}\cdot\hat{\mvec{n}}\;\mathrm{d}A = \int_{\Omega} \nabla \cdot (\phi\mvec{u})\;\mathrm{d}V
\end{equation}
and so \ref{eq:reynoldstransportheromem} becomes:
\begin{equation}
\frac{\mathrm{d}~}{\mathrm{d}t}\int_{\Omega} \phi\;\mathrm{d}V = -\int_{\Omega} \nabla \cdot (\phi\mvec{u})\;\mathrm{d}V + \int_{\Omega} s\;\mathrm{d}V
\end{equation}

\begin{assumption}
We assume $\phi$ is a continuous function both in space and time. So using Leinbiz's rules, the integral and derivative can rearranged.
\end{assumption}

Using the Leinbiz's rules, we have:
\begin{equation}
\int_{\Omega}  \frac{\partial}{\partial t}\phi\;\mathrm{d}V = -\int_{\Omega} \nabla \cdot (\phi\mvec{u})\;\mathrm{d}V + \int_{\Omega} s\;\mathrm{d}V
\end{equation}
and by combining the terms under the same integral simplifies to:
\begin{equation}
\int_{\Omega}  \left(\frac{\partial}{\partial t}\phi + \nabla \cdot (\phi\mvec{u}) - s\right)\;\mathrm{d}V = 0
\end{equation}
Therefore, the follow equation must also satisfy:
\begin{equation}
\frac{\partial \phi}{\partial t} + \nabla \cdot (\phi\mvec{u}) = s
\end{equation}

The resulting form in the well-known conservation form (or divergence form). Note that the second term in the LHS is a dyad or outer product, i.e.:
\begin{equation}
\phi {\mvec{u}} \equiv {\phi _i}{\mvec{u}_j} = \left( {\begin{array}{*{20}{c}}
{{\phi _1}}\\
{{\phi _2}}\\
{{\phi _3}}
\end{array}} \right)\left( {\begin{array}{*{20}{c}}
{{u_1}}&{{u_2}}&{{u_3}}
\end{array}} \right) = \left( {\begin{array}{*{20}{c}}
{{\phi _1}{u_1}}&{{\phi _1}{u_2}}&{{\phi _1}{u_3}}\\
{{\phi _2}{u_1}}&{{\phi _2}{u_2}}&{{\phi _2}{u_3}}\\
{{\phi _3}{u_1}}&{{\phi _3}{u_2}}&{{\phi _3}{u_3}}
\end{array}} \right)
\end{equation}
where the product of two rank-1 tensor (\textit{vector}) results in a rank-2 tensor (\textit{tensor}).

\newpage

\section{Conservation of mass}
\label{sec:conservationofmass}

The mass of moist air $m$ (kg) is assumed to consist of dry air $m_a$ (kg), water vapor $m_v$ (kg) and carbon-dioxide (i.e., CO$_2$) $m_{c}$ (kg).
\begin{equation}
m \equiv \sum_i m_i = m_a + m_v + m_{c}
\end{equation}

\begin{assumption}
	We do not model oxygen concentration change due to photosynthesis, i.e., $m_o = \textit{constant}$.
\end{assumption}

\subsection{Deriving conservation of mass}

Applying the Reynolds transport theorem for mass, we can derive the conservation of mass in the fluid domain:

\begin{quote}
	\centering
	Rate of increase of \textit{mass} in domain $\Omega$\\
	$=$\\
	Net rate of transfer of \textit{density} by \textit{advection} at boundary $\partial \Omega$\\
	$+$ \\
	Net rate of transfer of \textit{density} by \textit{diffusion} at boundary $\partial \Omega$\\
	$+$\\
	Net rate of transfer of \textit{density} within the domain $\Omega$ (source/sink)
\end{quote}

The net mass of gas mixture in the domain is given as:
\begin{equation}
m = \int_\Omega  \rho \:\mathrm{d}V
\end{equation}
where $\rho$ (kg\,m$^{-3}$) is density of the gas mixture. The conservation principle also applies for each individual species $i$ is given as:
\begin{equation}
{m_i} = \int_\Omega  {{\rho _i}\:\mathrm{d}V}
\end{equation}

The conservation of mass of individual species $i$ is given as:
\begin{equation}
\frac{\mathrm{d}~}{\mathrm{d}t} {m_i} = \frac{\mathrm{d}~}{\mathrm{d}t} \int \limits_{\Omega} \rho_i\;\mathrm{d}V= - \int \limits_{\partial \Omega } \mvec{g}_i\cdot\hat{\mvec{n}}\;\mathrm{d}A + \int \limits_{\Omega} s_{\rho,i}\;\mathrm{d}V
\end{equation}
where $\mvec{g}_i$ (kg\,m$^{-2}$\,s$^{-1}$) is the mass flux of species $i$ at the boundary, and $s_{\rho,i}$ (kg\,m$^{-3}$\,s$^{-1}$) is the source of mass in domain $\Omega$. The rate of loss or gain of gas mixture at the boundary of the domain is sum of convection due bulk fluid motion and diffusion resulted by the concentration gradient. The net flux of density from the control volume $\Omega$ is a combined convection-diffusion equation:
\begin{equation}
\int \limits_{\partial \Omega } {{{\mvec{g}}_i}}  \cdot \hat{\mvec{n}}\:\mathrm{d}A = \int \limits_{\partial \Omega } {\left( { - {\rho}{D}\nabla \frac{{{\rho _i}}}{{{\rho}}} + {\rho _i}{\mvec{u}}} \right)}  \cdot \hat{\mvec{n}}\:\mathrm{d}A
\label{eq:masscde}
\end{equation}
where $\mvec{g}_i$ (kg\,m$^{-2}$\,s$^{-1}$) is the mass flux of species $i$, $x_i \equiv \rho_i/\rho$ (kg\,kg$^{-1}$) is the mass concentration of species $i$, $D$ is the mass diffusivity (m$^2$s) of species $i$ and $\mvec{u}$ is the bulk / mass-averaged velocity:
\begin{equation}
{\mvec{u}} = \frac{{{\Sigma _i}{\rho _i}{{\mvec{u}}_i}}}{{{\Sigma _i}{\rho _i}}}
\end{equation}

The divergence theorem transforms \ref{eq:masscde} into:
\begin{equation}
\int \limits_{\partial \Omega } {{{\mvec{g}}_i}}  \cdot \hat{\mvec{n}}\:\mathrm{d}A = \int \limits_\Omega  \nabla  \cdot {{\mvec{g}}_i}\;\mathrm{d}V = \int \limits_\Omega  \nabla  \cdot \left( { - \rho D\nabla \frac{{{\rho _i}}}{{{\rho }}} + {\rho _i}{\mvec{u}}} \right)\;\mathrm{d}V
\end{equation}

The resulting conservation of mass for individual species is given as:
\begin{equation}
\frac{{\partial {\rho _i}}}{{\partial t}} + \nabla  \cdot \left( {{\rho _i}{\mvec{u}} - {\rho}{D}\nabla \frac{{{\rho _i}}}{{{\rho}}}} \right) = {s_{\rho ,i}}
\end{equation}

We have following system of equation for quaternary mixture of dry air, water vapor and CO$_2$:
\begin{align}
\frac{{\partial {\rho _a}}}{{\partial t}} + \nabla  \cdot \left( {{\rho _a}{\mvec{u}} - {\rho }{D}\nabla \frac{{{\rho _a}}}{{{\rho }}}} \right) &= {s_{\rho ,a}} \label{eq:comaa}\\
\frac{{\partial {\rho _v}}}{{\partial t}} + \nabla  \cdot \left( {{\rho _v}{\mvec{u}} - {\rho }{D}\nabla \frac{{{\rho _v}}}{{{\rho }}}} \right) &= {s_{\rho ,v}} \label{eq:comvv}\\
\frac{{\partial {\rho _{c}}}}{{\partial t}} + \nabla  \cdot \left( {{\rho _{c}}{\mvec{u}} - {\rho }{D}\nabla \frac{{{\rho _{c}}}}{{{\rho }}}} \right) &= {s_{\rho ,{c}}}
\end{align}
where ${s_{\rho ,a}}$, ${s_{\rho ,v}}$ and ${s_{\rho ,c}}$ are the mass source terms (kg\,m$^{-3}$s$^{-1}$).

\subsection{Source of mass}

\begin{assumption}
	We assume there is no dry air generated in the fluid, ${s_{\rho ,a}} = 0$. 
\end{assumption}

\begin{assumption}
	We assume that the source of water vapor is only due to leaf transpiration, i.e., $s_{\rho,v} = g_{\textit{v,leaf}}$. Water vapor condensation to water droplets or droplet evaporation and sublimation from ice is neglected.
\end{assumption}

The leaves in control volumes $\Omega$ generate water vapour (from the transpiration process) and extract $CO_2$ during photosynthesis. Therefore, the source of water vapour and $CO_2$ in control volume is:
\begin{align}
{s_{\rho ,v}} &= a\, g_{v,leaf} \label{eq:gvsource}\\
{s_{\rho ,c}} &= a\, g_{c,leaf}
\end{align}
where $a$ is the leaf area density (m$^2$m$^{-3}$), and $g_{v,\mathit{leaf}}$ (kg\,m$^{-2}$s$^{-1}$) and $g_{c,\mathit{leaf}}$ (kg\,m$^{-2}$s$^{-1}$) are the water vapour and $CO_2$ mass flux from the surface of the leaf.

Thus the conservation of mass of gas mixture is written as:
\begin{equation}
\frac{{\partial {\rho}}}{{\partial t}} + \nabla  \cdot \left( {{\rho }{\mvec{u}}} \right) = s_{\rho,v} + s_{\rho,c}
\end{equation}




\begin{assumption}
The order of magnitude of water vapor mass source $\mathcal{O}\left(s_{\rho,v}\right)\approx10^{-4}$ and $CO_2$ mass source $\mathcal{O}\left(s_{\rho,c}\right)\approx10^{-6}$ \citep{Hiraoka2005}. Therefore the mass source of $CO_2$ is negligible compared to water vapour.
\end{assumption}

\begin{assumption}
The continuity equation is used to solve the momentum equation. In the momentum equation, the momentum contribution due to the mass source can be assumed to be negligible w.r.t to the drag force terms.
\end{assumption}

Therefore, when solving the Navier-Stokes equations, the conservation of mass can be simplified to
\begin{equation}
\frac{{\partial {\rho}}}{{\partial t}} + \nabla  \cdot \left( {{\rho }{\mvec{u}}}\right) = 0
\end{equation}

\section{Conservation of momentum}
\label{sec:conservationofmomentum}

\subsection{Netwon's second law of motion}
Newton's second law of motion is given as:
\begin{equation}
{\mvec{F}} = \frac{\mathrm{d}~}{{\mathrm{d}t}}m{\mvec{u}}=\frac{\mathrm{d}~}{{\mathrm{d}t}} \int \limits_\Omega  \rho {\mvec{u}}\;\mathrm{d}V
\end{equation}
where $\mvec{F}$ (N) is the net force. The forces that act on the fluid are body forces (directly on volumetric mass) such as gravitational, electric, magnetic and if we have vegetation if modelled as porous media (source/sink). The vegetation is not directly modeled, but their contribution to the fluid is taken as source and sink terms in conservation equations. The surface forces are acting on the surface of fluid, such as pressure force, normal and shear stresses on the surface. These contribution can be regarded influences from neighboring fluid parcels.

\subsection{Deriving conservation of momentum}

Conservation of momentum is given as: 
\begin{quote}
	\centering
	Rate of increase of \textit{momentum} in $\Omega$\\
	$=$\\
	Net rate of transfer of \textit{momentum} by \textit{advection} at the boundary $\partial \Omega$\\
	$+$\\
	Net rate of transfer of \textit{momentum} by \textit{diffusion} at the boundary $\partial \Omega$\\
	$+$\\
	Net rate of transfer of \textit{momentum} by \textit{external forces} on domain $\Omega$\\
	$+$\\
	Net rate of transfer of \textit{momentum} within $\Omega$ (source/sinks)
\end{quote}

In addition to the convective loss of momentum at the boundary, in Reynolds transport theorem, stress at the boundary of the domain results in loss of momentum. The Cauchy stress tensor $\bar{\bar{\sigma}}$ (Pa or N\,m$^{-2}$ or kg\,m$^{-1}$\,s$^{-2}$) is defined as:
\begin{equation}
\bar{\bar{\sigma}} = \left( {\begin{array}{*{20}{c}}
	{{\sigma _{11}}}&{{\sigma _{12}}}&{{\sigma _{13}}}\\
	{{\sigma _{21}}}&{{\sigma _{22}}}&{{\sigma _{23}}}\\
	{{\sigma _{31}}}&{{\sigma _{32}}}&{{\sigma _{33}}}
	\end{array}} \right) = \left( {\begin{array}{*{20}{c}}
	{{\sigma _x}}&{{\tau _{xy}}}&{{\tau _{xz}}}\\
	{{\tau _{yx}}}&{{\sigma _y}}&{{\tau _{yz}}}\\
	{{\tau _{zx}}}&{{\tau _{zy}}}&{{\sigma _z}}
	\end{array}} \right)
\end{equation}
where the diagonal terms are normal stresses and the non-diagonal terms are the shear stresses. 



The resulting Reynolds transport theorem for momentum equation of a gas mixture, 
\begin{equation}
{\mvec{F}} = \frac{\mathrm{d}~}{{\mathrm{d}t}} \int \limits_\Omega  {\rho }{{\mvec{u}}}\;\mathrm{d}V =  - \int \limits_{\partial\Omega}  \left( \rho \mvec{u}\mvec{u} - \bar{\bar{\sigma}}\right) \cdot \hat{\mvec{n}}\;\mathrm{d}A + \int \limits_\Omega \mvec{f}\; \mathrm{d}V + \int \limits_\Omega \mvec{s}_{u}\; \mathrm{d}V
\end{equation}

Applying the divergence theorem, we get
\begin{equation}
\frac{\mathrm{d}~}{{\mathrm{d}t}} \int \limits_\Omega  {\rho }{{\mvec{u}}}\;\mathrm{d}V =  - \int \limits_{\Omega}  \nabla \cdot \left(\rho \mvec{u} \mvec{u} \right)\; \mathrm{d}V +  \int \limits_\Omega \nabla \cdot \bar{\bar{{\sigma}}}\;\mathrm{d}V + \int \limits_\Omega \mvec{f}\; \mathrm{d}V + \int \limits_\Omega \mvec{s}_{u}\; \mathrm{d}V
\end{equation}

And combining under same integral, assuming momentum is continuous in space and time:
\begin{equation}
\int \limits_\Omega \left[ \frac{\partial}{\partial t} \left(\rho \mvec{u} \right) + \nabla \cdot \left(\rho \mvec{u}\mvec{u} \right) - \nabla \cdot \bar{\bar{{\sigma}}} - \mvec{f} -  \mvec{s}_{u} \right]\; \mathrm{d}V = 0
\end{equation}

Therefore, the following relationship also satisfies (N\,m$^{-3}$):
\begin{equation}
\frac{\partial}{\partial t} \left(\rho \mvec{u} \right) + \nabla \cdot \left(\rho \mvec{u}\mvec{u} \right) = \nabla \cdot \bar{\bar{{\sigma}}} +  \mvec{f} +  \mvec{s}_{u}
\end{equation}

The cauchy stress tensor $\bar{\bar{{\sigma}}}$ (N\,m$^{-2}$) can be decomposed into the deviatoric and the hydrostatic component:
\begin{equation}
\bar{\bar{{\sigma}}} =  \underbrace{ \frac{1}{3}\mathrm{tr}\left(\bar{\bar{{\sigma}}}\right) \textbf{I} }_\text{hydrostatic}  + \underbrace{ \bar{\bar{{\sigma}}} - \frac{1}{3}\mathrm{tr}\left(\bar{\bar{{\sigma}}} \right)\textbf{I} }_\text{deviatoric}
\end{equation}

The hydrostatic stress component is the isotropic pressure, $\mathrm{tr}\left(\bar{\bar{{\sigma}}} \right)/3 = -p$ and the deviatoric component is the shear-stress tensor $\bar{\bar{{\tau}}}$. Thus, the Cauchy stress tensor becomes:
\begin{equation}
\bar{\bar{\sigma}} = -p\textbf{I} + \bar{\bar{{\tau}}}
\end{equation}

\begin{assumption}
	We assume Newtonian fluid, and therefore the viscous shear stress is symmetric and assumed to be linearly proportional to the local strain rate and equivalently the velocity gradient, i.e., $\tau_{\textit{ij}} \propto \frac{1}{2}\left(\pde{u_i}{x_j} + \pde{u_j}{x_i}\right)$.
\end{assumption}

Applying the Newtonian fluid hypothesis:
\begin{equation}
\bar{\bar{{\tau}}} = \mu \left( \nabla \mvec{u} + \left(\nabla \mvec{u} \right)^T - \frac{2}{3} \left(\nabla \cdot \mvec{u} \right) \textbf{I} \right) + \lambda \left(\nabla \cdot \mvec{u} \right)\textbf{I}
\end{equation}
where $\mu$ (N\,s\,m$^{-2}$ or kg\,m$^{-1}$\,s$^{-1}$) is the first coefficient of viscosity (i.e. dynamic viscosity), and $\lambda$ (N\,s\,m$^{-2}$) is the second coefficient of viscosity. Thus:
\begin{equation}
\begin{split}
\frac{\partial}{\partial t} \left(\rho \mvec{u} \right) &+ \nabla \cdot \left(\rho \mvec{u}\mvec{u} \right) = \\
&\nabla \cdot \left[ -p\textbf{I} + \mu \left( \nabla \mvec{u} + \left(\nabla \mvec{u} \right)^T - \frac{2}{3} \left(\nabla \cdot \mvec{u} \right) \textbf{I} \right) + \lambda \left(\nabla \cdot \mvec{u} \right)\textbf{I}\right]\\
& + \mvec{f} +  \mvec{s}_{u}
\end{split}
\end{equation}
and so:
\begin{equation}
\begin{split}
\frac{\partial}{\partial t} \left(\rho \mvec{u} \right) &+ \nabla \cdot \left(\rho \mvec{u}\mvec{u} \right) = - \nabla p + \nabla \cdot \left[ \mu \left( \nabla \mvec{u} + \left(\nabla \mvec{u} \right)^T \right) \right] \\
& + \nabla \cdot \left[ \left( \lambda - \frac{2}{3}\mu \right) \left(\nabla \cdot \mvec{u} \right) \textbf{I} \right] +  \mvec{f} +  \mvec{s}_{u}
\end{split}
\end{equation}

\begin{assumption}
We assume the only body forces are due to gravitational acceleration.
\end{assumption}

\begin{equation}
\begin{split}
\frac{\partial}{\partial t} \left(\rho \mvec{u} \right) &+ \nabla \cdot \left(\rho \mvec{u}\mvec{u} \right) = - \nabla p + \nabla \cdot \left[ \mu \left( \nabla \mvec{u} + \left(\nabla \mvec{u} \right)^T \right) \right] \\
& + \nabla \cdot \left[ \left( \lambda - \frac{2}{3}\mu \right) \left(\nabla \cdot \mvec{u} \right) \textbf{I} \right] +  \rho \mvec{g} +  \mvec{s}_{u}
\end{split}
\end{equation}

\begin{assumption}
	If we assume divergence-free velocity, i.e., $\nabla \cdot \mvec{u} = 0$, the shear-stress tensor is simply: $\bar{\bar{{\tau}}} = \mu \left( \nabla \mvec{u} + \left(\nabla \mvec{u} \right)^T \right)$.
\end{assumption}

\subsection{Source of momentum}

The source term due to vegetation, i.e. $\mvec{s}_u$ (kg\,m$^{-2}$\,s$^{-1}$), the volumetric force exerted by vegetation is modeled through Darcy-Forchheimer law for the flow through porous medium:
\begin{equation}
\mvec{s}_u =  - \left( {\left[ {\frac{{{\mu}\;\;}}{K}} \right]{\mvec{u}} + {\rho}\left[ {\frac{{{C_F}}}{{\;\sqrt K \;}}} \right]\left| {\mvec{u}} \right|{\mvec{u}}} \right)
\label{eq:darcyforchheimer}
\end{equation}
where $K$ (m$^{2}$) is permeability, $C_F$ is the Forchheimer coefficient for non-linear momentum loss \citep{Verboven2006,Boulard2008}.

\begin{assumption}
	For high wind speeds, that the Forcheimer term becomes the dominant influence due to the quadratic relation. Furthermore, the flow in vegetation is this regime. Therefore, we assume the linear Darcy term is negligible. 
\end{assumption}

The Forchheimer term for flow past porous screens (or windbreaks) with thickness $\delta t$ (m) is: 
\begin{equation}
\frac{{{C_F}}}{{\sqrt {{k_p}} }} = \frac{{{C_D}}}{{\delta t}}
\end{equation}
For vegetation the non-linear term is expressed in terms of leaf area density $a$ (m$^2_{\textit{leaf}}$\,m$^{-3}$ $\rightarrow$ m$^{-1}$) and leaf drag coefficient $c_d$:
\begin{equation}
\frac{{{C_F}}}{{\sqrt {{k_p}} }} = c_d\,a
\end{equation}
and the source term of vegetation as porous medium becomes \citep{Wilson1977,Liu1996,Hiraoka,Kenjeres2013}:
\begin{equation}
\mvec{s}_u = - \rho c_d a |\mvec{u}| \mvec{u}
\end{equation}

Thus, the momentum transport equation becomes:
\begin{equation}
\frac{\partial}{\partial t} \left(\rho \mvec{u} \right) + \nabla \cdot \left(\rho \mvec{u}\mvec{u} \right) = - \nabla p + \nabla \cdot \left[ \mu \left( \nabla \mvec{u} + \left(\nabla \mvec{u} \right)^T \right) \right] + \rho \mvec{g} - \rho c_d a |\mvec{u}| \mvec{u}
\end{equation}

%\begin{equation}
%\begin{split}
%\frac{\partial}{\partial t} \left(\rho \mvec{u} \right) &+ \nabla \cdot \left(\rho \mvec{u}\mvec{u} \right) = - \nabla p + \nabla \cdot \left[ \mu \left( \nabla \mvec{u} + \left(\nabla \mvec{u} \right)^T \right) \right] \\
%& + \nabla \cdot \left[ \left( \lambda - \frac{2}{3}\mu \right) \left(\nabla \cdot \mvec{u} \right) \textbf{I} \right] +  \rho \mvec{g} +  \mvec{s}_{u}
%\end{split}
%\end{equation}

\subsection{Deriving drag coefficient}

When performing measurements, the drag coefficient is characterized w.r.t to the frontal surface area of vegetation $A_{\textit{frontal}}$. The measured drag coefficient $C_d$ is related to the leaf drag coefficient $c_d$ from the following relationship: 
The total force on vegetation $F_d$ (N) is given as:	 
\begin{equation}
F_d = \frac{1}{2}{\rho}U_\infty ^2{C_d}{A_{\textit{frontal}}}
\label{eq:fd1}
\end{equation}
and should be equal to the volumetric force, i.e.:
\begin{equation}
F_d = - \mathop \int\limits_\Omega \mvec{s}_u\;\mathrm{d}V = \mathop \int \limits_\Omega  {\rho}{c_d}{a} |\mvec{u}|\mvec{u}\;\mathrm{d}V
\label{eq:fd2}
\end{equation}

\begin{assumption}
We assume $\rho = \textit{constant}$ and $c_d = \textit{constant}$.
\end{assumption}
So, equation \cref{eq:fd1} to \cref{eq:fd1} and substituting $U_\infty$ into $\mvec{u}$, we get:
\begin{equation}
\frac{1}{2}{\rho}U_\infty ^2{C_d}{A_{\textit{frontal}}} = {\rho}{c_d} U_\infty ^2 \mathop \int \limits_\Omega {a}  \;\mathrm{d}V
\label{eq:fd3}
\end{equation}
and so, the leaf drag coefficient becomes:
\begin{equation}
c_d = C_d\frac{A_{\textit{frontal}}}{2A_{\textit{leaf}}}
\end{equation}
as the integral of leaf area density $a$ (m$^{2}$\,m$^{-2}$) is:
\begin{equation}
\mathop \int\limits_\Omega a\;\mathrm{d}V = A_{\textit{leaf}}
\end{equation}
where $A_{\textit{leaf}}$ (m) is the total leaf surface area.


\newpage
\section{Conservation of Energy}
\label{sec:conservationofenergy}

\subsection{First law of thermodynamics}

First law of thermodynamics:
\begin{equation}
dE = \delta Q + \delta W
\end{equation}
where $dE$ (J or kg\,m$^2$\,s$^{-2}$) is the change in total internal energy of the system, $\delta Q$ (J) is the heat added to the system, and $\delta W$ (J) is the work done on the system. The total energy per unit mass of the mixture $\hat{E}$ is defined as:
\begin{equation}
E = \int\limits_\Omega \rho \hat{E}\;\mathrm{d}V
\end{equation}
where $\hat{E}$ (J\,kg$^{-1}$) (an intensive property) is related total energy $E$ (J) (an extensive property). The total energy per unit mass of the mixture $\hat{E}$ is given as: 
\begin{equation}
\hat{E} = e + \frac{\left| \mvec{u} \right|^2}{2} + gz
\end{equation}
where $e$ (J\,kg$^{-1}$) is the internal energy per unit mass, $|\mvec{u}|^2/2$ (J\,kg$^{-1}$) is the kinetic energy per unit mass, and $gz$ (J\,kg$^{-1}$) is the potential energy per unit mass. The internal energy of a gas mixture can be related to enthalpy and kinetic theory of gas:
\begin{equation}
e = h - RT = h - \frac{p}{\rho}
\end{equation}
where $R$ (J\,kg$^{-1}$\,K$^{-1}$) is the gas constant. The total enthalpy $h$ (J\,kg$^{-1}$) of the gas mixture is defined in \cref{app:thermodynamics}, i.e., \cref{eq:enthalpymoistair} and given as:
 \begin{equation}
h = \sum_i x_i h_i =  \left(x_a c_{pa} + x_v c_{pv} \right) \left(T - T_{\textit{ref}}\right)  + x_v\,L_v 
\label{eq:enthalpymoistair2}
\end{equation}

Therefore, total energy (per unit mass) of the gas mixture $\hat{E}$ (J\,kg$^{-1}$) is:
\begin{equation}
\hat{E} = \sum_i x_i h_i - \frac{p}{\rho} + \frac{\left| \mvec{u} \right|^2}{2} + gz
\label{eq:energysimple}
\end{equation}
and substituting \cref{eq:enthalpymoistair2} into \cref{eq:energysimple}:
\begin{equation}
\hat{E} = \left(x_a c_{pa} + x_v c_{pv} \right) \left(T - T_{\textit{ref}}\right)  + x_v L_v - \frac{p}{\rho} + \frac{\left| \mvec{u} \right|^2}{2} + gz
\label{eq:energysimple2}
\end{equation}

\subsection{Deriving conservation of energy}

The conservation of energy is given as:
 \begin{quote}
	\centering
	Rate of change of total \textit{energy} in domain $\Omega$\\
	$=$\\
	Rate of transfer of \textit{energy} by \textit{advection} at boundary $\partial \Omega$\\
	$+$\\		
	Rate of transfer of \textit{heat} by \textit{heat diffusion} at boundary $\partial \Omega$\\
	$+$\\
	Rate of transfer of \textit{heat} by \textit{mass diffusion} at boundary $\partial \Omega$\\
	$+$\\
	Rate at which \textit{work} done by boundary $\Omega$\\	
	$+$\\	
	Rate of transfer of \textit{heat} by within $\Omega$ (source/sink)
\end{quote}

Therefore, the conservation of energy (J\,s$^{-1}$ or W) is given as:
\begin{equation}
\frac{\mathrm{d}~}{{\mathrm{d}t}}E = \frac{\mathrm{d}~}{{\mathrm{d}t}} \int \limits_\Omega  \rho \hat{E}\;\mathrm{d}V
\end{equation}
and applying the Reynolds transport theorem, detailed in \cref{sec:Conservationprinciple}, we attain conservation of energy (W\,m$^{-3}$) as:
\begin{equation}
\frac{\partial }{{\partial t}}\left( {{\rho }{\hat{E}}} \right) + \nabla  \cdot \left( {{\rho }{\hat{E}}{\mvec{u}}} \right) =  - \nabla  \cdot {\mvec{q}} - \nabla  \cdot \left( {{p}{\mvec{u}}} \right) - \nabla  \cdot \left(  \bar{\bar{{\tau}}} \cdot \mvec{u} \right) + s_h
\label{eq:energy1}
\end{equation}
where $\mvec{q}$ is the total heat flux due to heat diffusion and mass diffusion, $\nabla \cdot \left(p\mvec{u}\right)$ (W\,m$^{-3}$ or J\,m$^{-3}$\,s$^{-1}$ or kg\,m$^{-3}$\,s$^{-1}$) is the work done due to pressure force, $\nabla  \cdot \left(  \bar{\bar{{\tau}}} \cdot \mvec{u} \right)$ (W\,m$^{-3}$) is the work done due to viscous force, and $s_h$ (W\,m$^{-3}$) rate of energy added into the system, i.e. the energy source. 

\begin{assumption}
We assume pressure work is negligible due to incompressible flow approximation, and viscous heat is neglected as well \citep{Defraeye2011}. Furthermore, the source of energy is simply due to radiation (at walls) and from vegetation. 
\end{assumption}

The total heat flux $\mvec{q}$ (W\,m$^{-2}$) is the total heat flux due to heat diffusion (i.e., conduction) and mass diffusion is given as:
\begin{equation}
\mvec{q} = \mvec{q}_c + \mvec{q}_d
\label{eq:totalheatflux}
\end{equation}
where $\mvec{q}_c$ (W\,m$^{-2}$) is the heat flux due to heat diffusion (i.e., conduction) and $\mvec{q}_d$ (W\,m$^{-2}$) is the heat flux due to mass diffusion. The conduction term is defined by the Fourier law of heat conduction:
\begin{equation}
\mvec{q}_c = - \lambda \nabla T
\end{equation}
where $\lambda = c_p\mu/\mathrm{Pr}$ (W\,m$^{-1}$\,K$^{-1}$) is the thermal conductivity, where $c_p$ (J\,kg$^{-1}$\,K$^{-1}$) is the specific heat capacity of gas mixture, $\mu$ (kg\,m$^{-1}$\,s$^{-1}$) is the dynamic viscosity of gas mixture. The heat flux due to mass flux $\mvec{q}_d$ (W\,m$^{-2}$) is given as:
\begin{equation}
\mvec{q}_d = \sum_i h_i\, \mvec{g}_{\textit{d,i}}
\end{equation}
where $\mvec{g}_{\textit{d,i}}$ (kg\,m$^{-2}$\,s$^{-1}$) diffusive flux of species $i$ contributing to the heat transport. So:
\begin{equation}
\mvec{q} = - \lambda \nabla T + \sum_i h_i\, \mvec{g}_{\textit{d,i}}
\end{equation}

Simplifying \cref{eq:energy1}, we obtain:
\begin{equation}
\frac{\partial }{{\partial t}}\left( {{\rho }{\hat{E}}} \right) + \nabla  \cdot \left( {{\rho }{\hat{E}}{\mvec{u}}} \right) =  - \nabla  \cdot {\mvec{q}}  + s_h
\label{eq:energy1p1}
\end{equation}


%\begin{assumption}
%	We assume, evaporation (note: don't confuse evaporation to transpiration), condensation, sublimation, or melting does not occur in the domain. Therefore, their source of energy is zero.  
%\end{assumption}


We expand \ref{eq:energy1p1} by substituting total energy per unit mass $\hat{E}$ (J\,kg$^{-1}$), \ref{eq:energysimple}, giving:
\begin{equation}
\begin{split}
&\frac{\partial }{\partial t} \left\{\rho \left( \sum_i x_i h_i  - \frac{p}{\rho} + \frac{\left| \mvec{u} \right|^2}{2} + gz  \right) \right\}\\
& \hspace{5em} + \nabla  \cdot \left\{ \rho \left(   \sum_i x_i h_i  - \frac{p}{\rho} + \frac{\left| \mvec{u} \right|^2}{2} + gz  \right) \mvec{u} \right\} \\
& \hspace{10em}  =  - \nabla  \cdot \mvec{q} + s_h
\end{split}
\end{equation}


and taking $\rho$ inside and substituting $\rho_i = \rho x_i $ gives:
\begin{equation}
\begin{split}
&\frac{\partial }{\partial t} \left( \sum_i \rho_i h_i  - p + \frac{\rho \left| \mvec{u} \right|^2}{2} + \rho gz  \right) \\
& \hspace{5em} + \nabla  \cdot \left\{ \left(   \sum_i \rho_i h_i  - p + \frac{\rho \left| \mvec{u} \right|^2}{2} + \rho gz  \right) \mvec{u} \right\} \\
& \hspace{10em} =  - \nabla  \cdot \mvec{q} + s_h
\end{split}
\label{eq:energy2}
\end{equation}

\begin{assumption}
	We assume pressure, potential energy and kinetic energy variation is small. 
\end{assumption}	

Therefore, \cref{eq:energy2} further simplifies to:
\begin{equation}
\frac{\partial }{\partial t} \left( \sum_i \rho_i h_i \right) +   \nabla  \cdot \left( \left( \sum_i \rho_i h_i \right)\mvec{u}\right) =  - \nabla  \cdot \mvec{q} + s_h
\label{eq:energy3}
\end{equation}

Substituting enthalpies into \cref{eq:energy3}, the conservation of moist air becomes:
\begin{equation}
\begin{split}
&\frac{\partial }{\partial t} \left( \rho_a\,c_{\textit{pa}}\left(T - T_{\textit{ref}}\right) +  \rho_v\,c_{\textit{pv}} \left(T - T_{\textit{ref}}\right) + \rho_v\, L_v \right)\\
&  \hspace{5em} + \nabla  \cdot \left( \left( \rho_a\,c_{\textit{pa}}\left(T - T_{\textit{ref}}\right) +  \rho_v\,c_{\textit{pv}} \left(T - T_{\textit{ref}}\right) + \rho_v\, L_v \right) \mvec{u}\right)\\
& \hspace{10em} =  - \nabla  \cdot \mvec{q} + s_h
\end{split}
\label{eq:energy4}
\end{equation}

Substituting, the total heat flux $\mvec{q}$ (W\,m$^{-2}$) \cref{eq:totalheatflux}, into \cref{eq:energy4}, we obtain conservation equation of the form:
\begin{equation}
\begin{split}
&\frac{\partial }{\partial t} \left( \rho_a\,c_{\textit{pa}}\left(T - T_{\textit{ref}}\right) +  \rho_v\,c_{\textit{pv}} \left(T - T_{\textit{ref}}\right) + \rho_v\, L_v \right)\\
& \hspace{5em} + \nabla  \cdot \left( \left( \rho_a\,c_{\textit{pa}}\left(T - T_{\textit{ref}}\right) +  \rho_v\,c_{\textit{pv}} \left(T - T_{\textit{ref}}\right) + \rho_v\, L_v \right) \mvec{u}\right)\\
& \hspace{10em} =  - \nabla  \cdot \left(- \lambda \nabla T + \sum_i h_i \mvec{g}_{\textit{d,i}}\right) + s_h
\end{split}
\label{eq:energy5}
\end{equation}

and as $\rho_i\mvec{u}_i = \rho_i \mvec{u} + \mvec{g}_{\textit{d,i}}$ (see \cref{sec:conservationofmass}), \cref{eq:energy5} can be simplified to:
\begin{equation}
\begin{split}
&\frac{\partial }{\partial t} \left( \rho_a\,c_{\textit{pa}}\left(T - T_{\textit{ref}}\right) +  \rho_v\,c_{\textit{pv}} \left(T - T_{\textit{ref}}\right) + \rho_v\, L_v \right)\\
& \hspace{5em} + \nabla  \cdot \left( \rho_a\,c_{\textit{pa}}\left(T - T_{\textit{ref}}\right)\mvec{u}_a +  \rho_v\,c_{\textit{pv}} \left(T - T_{\textit{ref}}\right)\mvec{u}_v + \rho_v\, L_v\mvec{u}_v \right)\\
& \hspace{10em} =  \nabla  \cdot \left(\lambda \nabla T \right) + s_h
\end{split}
\label{eq:energy6}
\end{equation}
Therefore, the conservation of energy \cref{eq:energy6} can be rewritten for separate species as follows:
\begin{equation}
\begin{split}
c_{\textit{pa}}  &\left( \frac{\partial }{\partial t} \left( \rho_a T \right) + \nabla\cdot \left(\rho_a T \mvec{u}_a \right)  \right) + c_{\textit{pv}} \left( \frac{\partial }{\partial t} \left( \rho_v T \right) + \nabla \cdot\left(\rho_v T \mvec{u}_v \right)  \right) \\ 
- c_{\textit{pa}} T_{\textit{ref}} &\left( \frac{\partial }{\partial t} \left( \rho_a \right) + \nabla\cdot \left(\rho_a \mvec{u}_a \right)  \right) - c_{\textit{pv}} T_{\textit{ref}} \left( \frac{\partial }{\partial t} \left( \rho_v \right) + \nabla\cdot \left(\rho_v \mvec{u}_v \right)  \right) \\
 + L_v &\left( \frac{\partial }{\partial t} \left( \rho_v \right) + \nabla\cdot \left(\rho_v \mvec{u}_v \right)  \right) =  \nabla  \cdot \left(\lambda \nabla T \right) + s_h
\end{split}
\label{eq:energy7}
\end{equation}

\subsection{Source of energy}

The source of energy $s_h$ (W\,m$^{-3}$) is given defined as:
\begin{equation}
s_h = a\,\left(q_{\textit{c,leaf}} + q_{\textit{d,leaf}}\right)
\label{eq:shterm01}
\end{equation}
where $a$ (m$^{2}$\,m$^{-3}$) is leaf area density, $q_{\textit{c,leaf}}$ (W\,m$^{-2}$) is heat flux from leaf surface due to conduction (i.e., heat diffusion), and $q_{\textit{d,leaf}}$ (W\,m$^{-2}$) is the heat flux due to mass diffusion from the leaf surface. Expanding, the mass diffusion term in \cref{eq:shterm01} gives:
\begin{equation}
s_h = a\,\left(q_{\textit{c,leaf}} + \sum_i h_i g_{\textit{i,leaf}} \right)
\label{eq:shterm1}
\end{equation}
where $ g_{\textit{i,leaf}} $ is the mass flux of species $i$ from the leaf surface. And so \cref{eq:shterm1} becomes: 
\begin{equation}
s_h = a\, \left(q_{\textit{c,leaf}} + h_v g_{\textit{v,leaf}} + h_c g_{\textit{c,leaf}} \right)
\label{eq:shterm2}
\end{equation}
where $g_{\textit{v,leaf}}$ (kg\,m$^{-2}$\,s${-1}$) and $g_{\textit{c,leaf}}$ (kg\,m$^{-2}$\,s${-1}$) are mass fluxes of water vapor and CO$_2$, respectively.

\begin{assumption}
Note that, we assumed there is no flux of oxygen O$_2$.
\end{assumption}

The conductive heat flux $q_{\textit{c,leaf}}$ (W\,m$^{-2}$) is given as:
\begin{equation}
q_{\textit{c,leaf}} = h_{\textit{c,h}} \left(T_l - T\right)
\label{eq:qcleaf}
\end{equation}
where $h_{\textit{c,h}}$ (W\,m$^{-2}$\,K$^{-1}$) is the convective heat transfer coefficient of the leaf and $T_l$ (K) is the leaf temperature. The source of energy due to water vapor flux $g_{\textit{v,leaf}}$ is given as:
\begin{equation}
h_v g_{\textit{v,leaf}} = c_{\textit{pv}} \left(T_l - T\right) g_{\textit{v,leaf}} + L_v g_{\textit{v,leaf}}
\end{equation}
where $g_{\textit{v,leaf}}$ (kg\,m$^{-2}$\,s$^{-1}$) is the leaf transpiration rate. Similarly, the source of energy due to CO$_2$ flux $g_{\textit{c,leaf}}$ is given as:
\begin{equation}
h_c g_{\textit{c,leaf}} = c_{\textit{pc}} \left(T_l - T\right) g_{\textit{c,leaf}}
\label{eq:qdvleaf}
\end{equation}
where $g_{\textit{v,leaf}}$ (kg\,m$^{-2}$\,s$^{-1}$) is the leaf CO$_2$ assimilation rate (i.e., mass flux of CO$_2$ due to the photosynthetic process). 

Thus, substituting \cref{eq:qcleaf,eq:qdvleaf} into \cref{eq:shterm2}, and decompose to sensible (i.e., temperature dependent) and latent component, we arrive at:
\begin{equation}
{s_h} = a{\mkern 1mu} \left( {\underbrace {\left[ {{h_{\textit{c,h}}} + {c_{pv}}{g_{\textit{v,leaf}}} + {c_{pc}}{g_{\textit{c,leaf}}}} \right]\left( {{T_l} - T} \right)}_{\mathrm{sensible}} + \underbrace {{L_v}{g_{\textit{v,leaf}}}}_{\mathrm{latent}}} \right)
\label{eq:shterm3}
\end{equation}

\begin{assumption}
We assume the  sensible heat flux contribution of water vapor and CO$_2$ is negligible compared to the conductive heat flux \citep{Hiraoka2005}, i.e., $\left(c_{pv}g_{v,leaf} + c_{pc}g_{c,leaf} \right)\ll h_{\textit{c,h}}$.
\end{assumption}

Thus, the source of energy $s_h$ (W\,m$^{-3}$), \cref{eq:shterm3}, simplifies to:
\begin{equation}
{s_h} = a \left( h_{\textit{c,h}} \left(T_l - T\right) + L_v g_{\textit{v,leaf}} \right)
\label{eq:shterm4}
\end{equation}
where:
\begin{align}
	q_{\textit{sen,leaf}} &= h_{\textit{c,h}} \left(T_l - T\right) \label{eq:qsenleaf2} \\
	q_{\textit{lat,leaf}} &= L_v g_{\textit{v,leaf}} \label{eq:qlatleaf2} \\
\end{align}
$q_{\textit{sen,leaf}}$ (W\,m$^{-2}$) and $q_{\textit{lat,leaf}}$ (W\,m$^{-2}$) is typically known as the sensible and latent heat flux from leaf in literature \citep{Manickathan2018a,Hiraoka2005,Bruse1998}. 


Substituting \cref{eq:shterm4} into conservation of energy becomes:
\begin{equation}
\begin{split}
c_{\textit{pa}}  &\left( \frac{\partial }{\partial t} \left( \rho_a T \right) + \nabla\cdot \left(\rho_a T \mvec{u}_a \right)  \right) + c_{\textit{pv}} \left( \frac{\partial }{\partial t} \left( \rho_v T \right) + \nabla \cdot\left(\rho_v T \mvec{u}_v \right)  \right) \\ 
- c_{\textit{pa}} T_{\textit{ref}} &\left( \frac{\partial }{\partial t} \left( \rho_a \right) + \nabla \cdot\left(\rho_a \mvec{u}_a \right)  \right) - c_{\textit{pv}} T_{\textit{ref}} \left( \frac{\partial }{\partial t} \left( \rho_v \right) + \nabla \cdot\left(\rho_v \mvec{u}_v \right)  \right) \\
+ L_v &\left( \frac{\partial }{\partial t} \left( \rho_v \right) + \nabla\cdot \left(\rho_v \mvec{u}_v \right)  \right) =  \nabla  \cdot \left(\lambda \nabla T \right) + a \left( h_{\textit{c,h}} \left(T_l - T\right) + L_v g_{\textit{v,leaf}} \right)
\end{split}
\label{eq:energy8}
\end{equation}

Substituting conservation of mass of air \cref{eq:comaa} and conservation of mass of water vapor, \cref{eq:energy8} simplifies as:
\begin{equation}
\begin{split}
&c_{\textit{pa}}   \left( \frac{\partial }{\partial t} \left( \rho_a T \right) + \nabla\cdot \left(\rho_a T \mvec{u}_a \right)  \right) + c_{\textit{pv}} \left( \frac{\partial }{\partial t} \left( \rho_v T \right) + \nabla \cdot\left(\rho_v T \mvec{u}_v \right)  \right) \\ 
& \hspace{5em}- c_{\textit{pv}} T_{\textit{ref}} \left(a\,g_{\textit{v,leaf}}  \right) + L_v \left(a\,g_{\textit{v,leaf}}  \right) \\
& \hspace{10em} =  \nabla  \cdot \left(\lambda \nabla T \right) + a \left( h_{\textit{c,h}} \left(T_l - T\right) + L_v g_{\textit{v,leaf}} \right)
\end{split}
\label{eq:energy9}
\end{equation}

\begin{assumption}
We assume the term $c_{\textit{pv}} T_{\textit{ref}}$ is negligible, as  $c_{\textit{pv}} T_{\textit{ref}} \ll L_v$. 

\end{assumption}
\begin{equation}
\begin{split}
&c_{\textit{pa}}   \left( \frac{\partial }{\partial t} \left( \rho_a T \right) + \nabla \cdot \left(\rho_a T \mvec{u}_a \right)  \right) + c_{\textit{pv}} \left( \frac{\partial }{\partial t} \left( \rho_v T \right) + \nabla \cdot \left(\rho_v T \mvec{u}_v \right)  \right) \\ 
&\hspace{5em}+ L_v \left(a\,g_{\textit{v,leaf}}  \right) = \nabla  \cdot \left(\lambda \nabla T \right) + a \left( h_{\textit{c,h}} \left(T_l - T\right) + L_v g_{\textit{v,leaf}} \right)
\end{split}
\label{eq:energy10}
\end{equation}

Thus, we can cancel the latent component terms on both sides, giving:
\begin{equation}
\begin{split}
&c_{\textit{pa}}   \left( \frac{\partial }{\partial t} \left( \rho_a T \right) + \nabla \cdot \left(\rho_a T \mvec{u}_a \right)  \right) + c_{\textit{pv}} \left( \frac{\partial }{\partial t} \left( \rho_v T \right) + \nabla \cdot \left(\rho_v T \mvec{u}_v \right)  \right) \\ 
&\hspace{5em} = \nabla  \cdot \left(\lambda \nabla T \right) + a \left( h_{\textit{c,h}} \left(T_l - T\right) \right)
\end{split}
\label{eq:energy11}
\end{equation}

So, we see that following the assumption, the source of (sensible) energy (i.e., energy equation based on only the temperature) is simply due to sensible heat flux from the leaf surface. Thus, for such equations, the source of energy $s_h$ (W\,m$^{-3}$) is:
\begin{equation}
{s_h} =  a\,q_{\textit{sen,leaf}} = a\, h_{\textit{c,h}} \left(T_l - T\right)
\label{eq:shterm5}
\end{equation}

The energy equation (i.e., now the transport equation of temperature), \cref{eq:energy11} can be further simplified, as density of moist air $\rho = \rho_a + \rho_v$ (kg\,m$^{-3}$) and heat capacity $c_p = c_{\textit{pa}}x_a + c_{\textit{pv}}x_v$ (J\,kg$^{-1}$\,K$^{-1}$):
\begin{equation}
c_p  \left( \frac{\partial }{\partial t} \left( \rho T \right) + \nabla \cdot \left(\rho T \mvec{u} \right)  \right) = \nabla  \cdot \left(\lambda \nabla T \right) + \underbrace{a\,q_{\textit{sen,leaf}}}_{s_h}
\label{eq:energy12}
\end{equation}

