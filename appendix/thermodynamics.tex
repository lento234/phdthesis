\chapter{Thermodynamics of moist air}
\label{app:thermodynamics}


\section*{Ideal gas law}

The ideal gas law is defined as:
\begin{equation}
p = \rho\frac{\mathcal{R}}{M}T
\label{eq:univGasLaw}
\end{equation}
where $p$ is the pressure (Pa), $\rho$ is density (kg\,m$^{-3}$), $\mathcal{R}=\num{8.3145598}$ J\,mol$^{-1}$K$^{-1}$ is universal gas constant, $M$ is molar mass (kg\,mol$^{-1}$), and $T$ is temperature (K). The specific gas constant $R$ (J\,kg$^{-1}$K$^{-1}$) is defined as:
\begin{equation}
R = \frac{\mathcal{R}}{M}
\label{eq:R_to_univR}
\end{equation}
and substituting \cref{eq:R_to_univR} into \cref{eq:univGasLaw}, we get:
\begin{equation}
p = \rho R T
\label{eq:idealGasLaw}
\end{equation}
or similarly:
\begin{equation}
p_i = \rho_i R_i T_i
\label{eq:idealGasLaw2}
\end{equation}

\section*{Partial pressures}

Using Dalton's law of partial pressures, the pressure of moist air is assumed to be sum of partial pressure of dry air $p_a$ (Pa) and partial pressure of water vapor $p_v$ (Pa):
\begin{equation}
p = \sum_i p_i  = p_v + p_a
\label{eq:dalton}
\end{equation}

We assuming that all the species are in thermal equilibrium ($T = T_v = T_a$) (K) and so:
\begin{equation}
p = \left(\rho_a R_a + \rho_v R_v \right) T
\end{equation}

The molar mass of dry air and water vapor are $M_a = \num{18.0149}$ g\,mol$^{-1}$ and $M_v=\num{28.964}$ g\,mol$^{-1}$, respectively. The specific gas constant of dry air and water vapor are $R_a = \num{287.055}$ J\,kg$^{-1}$K$^{-1}$ and $R_v = \num{461.5}$ J\,kg$^{-1}$K$^{-1}$ , respectively \citep{ASHRAE2013}.

The vapor pressure $p_v$ is related to relative humidity $\phi$ as follows:
\begin{equation}
\frac{p_v}{p_{\textit{vsat}}} = \phi
\end{equation}
where $p_{\textit{vsat}}$ is the saturation vapor pressure. The relative humidity (RH) is also defined as $\textrm{RH}=\phi\times 100$. The saturation vapor pressure, $p_{\textit{vsat}}$ (Pa), is determined directly from temperature as follows \citep{Singh2002}:
%\begin{equation}
%p_{vsat} = 610.78 \exp \left\{\frac{17.269\ (T - 273.15)}{T - 35.85}\right\}
%\end{equation}
\begin{equation}
p_{vsat} = 610.78 \exp \left\{\frac{17.269\ (T - 273.15)}{T - 35.85}\right\} \qquad (0^{\circ}\mathrm{C} < t < 63^{\circ}\mathrm{C})
\end{equation}
where $t = T - 273.15$ ($^{\circ}$C) or similarly \citep{ASHRAE2013}:

\begin{align}
&p_{vsat} = \exp \left\{\frac{C_1}{T} + C_2 + C_3 T + C_4 T^2 + C_5 T^3 + C_6 \log(T)\right\}\\
& \hspace{10em} (0^{\circ}\mathrm{C} < t < 200^{\circ}\mathrm{C})
\end{align}
where $C_1 = \num{-5.8002206e3}$, $C_2 = \num{1.3914993e0}$, $C_3 = \num{-4.8640239e-2}$, $C_4 = \num{4.1764768e-5}$, $C_5 = \num{-1.4452093e-8}$, and $C_6 = \num{6.5459673e0}$.

\section*{Mass Fractions}

The mass of moist gas mixture is :
\begin{equation}
m = m_v + m_a
\end{equation}
where it is composed of water vapor $m_v$ (kg) and dry air $m_a$ (kg).

\section*{Mass concentration}

The mass concentration of species $i$, $x_i$ (kg$_i$\,kg$^{-1}$), is defined as:
\begin{equation}
x_i \equiv \frac{m_i}{m} = \frac{\rho_i}{\rho}
\end{equation}
where $m_i$ (kg$_i$) is the mass of species $i$, and $m$ (kg) is the total mass of the gas mixture. The mass fractions are related as follows:
\begin{equation}
\sum_i x_i = x_v + x_a = 1
\label{eq:totalconc}
\end{equation}

\section*{Specific humidity}

The specific humidity or mixing ratio $q$ (kg\,kg$^{-1}$) is defined as:
\begin{equation}
q \equiv \frac{m_v}{m} = \frac{\rho_v}{\rho}
\label{eq:def_q}
\end{equation}
where it the mass ratio of water vapor to the total mass. It also satisfies:
\begin{equation}
q = \frac{\rho - \rho_a}{\rho}
\label{eq:q_rho}
\end{equation}
and so solving for total density $\rho$, we obtain:
\begin{equation}
\rho = \frac{\rho_a}{1-q}
\label{eq:q_rho2}
\end{equation}

It is related to the humidity ratio $w$ (kg\,kg$^{-1}$) as follows:
\begin{equation}
q = \frac{w}{1 + w}
\label{eq:q_w}
\end{equation}

\section*{Humidity ratio}

The humidity ratio $w$ (kg\,kg$^{-1}$) (i.e., or moisture content or mixing ratio) is the ratio of mass of water vapor to dry air \citep{ASHRAE2013}:
\begin{equation}
w \equiv \frac{m_v}{m_a} = \frac{x_v}{x_a} = \frac{\rho_v}{\rho_a}
\label{eq:def_w}
\end{equation}
Substituting, \cref{eq:totalconc} into \cref{eq:def_w} we get
\begin{equation}
w = \frac{x_v}{1 - x_v}
\label{eq:w_xv}
\end{equation}

We can rewrite \cref{eq:w_xv} to determine $x_v$ from $w$:
\begin{equation}
x_v = \frac{w}{1 + w}
\label{eq:xv_w}
\end{equation}

We can determine $w$ from $p_v$:
\begin{equation}
w = \frac{p_v/R_v T}{p_a/ R_aT} = \frac{p_v}{p_a}\frac{R_a}{R_v}
\label{eq:w_to_P}
\end{equation}
and substituting \cref{eq:dalton} into \cref{eq:w_to_P}, we get:
\begin{equation}
w = \frac{p_v}{p-p_v}\frac{R_a}{R_v}
\label{eq:pv_to_w}
\end{equation}
and rewriting \ref{eq:pv_to_w} for $p_v$, we get:
\begin{equation}
p_v = \frac{p w}{R_a/R_v+w}
\label{eq:w_to_pv}
\end{equation}

\section*{Enthalpy}

The specific enthalpy $h$ (J\,kg$^{-1}$) of a unit mass is defined as:

\begin{equation}
h = e + \frac{p}{\rho}
\end{equation}

where $e$ (J\,kg$^{-1}$) is the specific internal energy per unit mass. 

\begin{assumption}
	We assume perfect gas.
\end{assumption}

For perfect gas, we have:
\begin{align}
\mathrm{d}e &= c_v \,\mathrm{d}T\\
\mathrm{d}h &= c_p \,\mathrm{d}T
\end{align}
where $c_v$ and $c_p$ are the specific heat capacity at constant volume and pressure, respectively.
\begin{assumption}
	We assume that the moist air is calorically perfect gas, i.e., $c_v=\textit{constant}$, $c_p=\textit{constant}$
\end{assumption}
The specific enthalpy is therefore:
\begin{equation}
h = c_p \int^T_{T_{\textit{ref}}} \mathrm{d}T = c_p \left(T - T_{\textit{ref}}\right)
\end{equation}
where we take $T_{\textit{ref}}$ at the reference temperature ($t=0$ \si{\degreeCelsius}).

\begin{assumption}
	We assume thermal equilibrium between all the mixtures, i.e., $T = T_a = T_v$ (K) \citep{Defraeye2011}.
\end{assumption}

\begin{assumption}
	We assume a binary gas mixture of dry air and water vapor. We assume CO$_2$ does not play a role in the conservation of energy and is simply a passive scalar.
\end{assumption}

Therefore, the specific enthalpy of moist air is 
\begin{equation}
h = \sum_i x_i h_i = x_a h_a + x_v h_v
\end{equation}
where $x_a$ and $x_v$ (kg\,kg$^{-1}$) are dry air and vapor mass concentration, and $h_a$ and $h_v$ (J\,kg$^{-1}$) are:
\begin{align}
h_a &= c_{pa} \left(T - T_{\textit{ref}}\right) \\
h_v &= c_{pv} \left(  T - T_{\textit{ref}}\right) + L_v
\end{align}
where $L_v$ (J\,kg$^{-1}$) is the latent heat of vaporization (more accurately, it is referenced at the same reference temperature as $T_{\textit{ref}}$). The reference temperature $T_{\textit{ref}}$ (K) is usually taken to be $t=0^{\circ}$C, i.e., $T_{\textit{ref}} = 273.15$ K and the latent heat of vaporization taken at the same temperature $L_v = \num{2.5e6}$ J\,kg$^{-1}$ at $t=0^{\circ}$C. Thus the specific heat of moist air $h$ (J\,kg$^{-1}$) becomes:
\begin{equation}
h = \left(x_a c_{pa} + x_v c_{pv} \right) \left(T - T_{\textit{ref}}\right)  + x_v L_v
\label{eq:enthalpymoistair}
\end{equation}
where the total specific heat capacity of gas $c_p$ (J\,kg\,$^{-1}$\,K$^{-1}$) is defined as:
\begin{equation}
c_p = x_a c_{pa} + x_v c_{pv}
\end{equation}

%\begin{assumption}
%Typically, for atmospheric flows with relatively low water vapor concentration, we may assume $x_v c_{pv} \ll x_a c_{pa}$ and so: 
%\end{assumption}
%
%\begin{equation}
%h \approx x_a c_{pa} \left(T - T_{\textit{ref}}\right)  + x_v L_v
%\end{equation}

