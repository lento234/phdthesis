\chapter{Thermodynamics of moist air}
\label{app:thermodynamics}


\section*{Ideal gas law}

The ideal gas law is defined as:
\begin{equation}
p = \rho\frac{\mathcal{R}}{M}T
\label{eq:univGasLaw}
\end{equation}2
where $p$ is the pressure (Pa), $\rho$ is density (kg\,m$^{-3}$), $\mathcal{R}=\num{8.3145598}$ J\,mol$^{-1}$K$^{-1}$ is universal gas constant, $M$ is molar mass (kg\,mol$^{-1}$) and $T$ is temperature (K). The specific gas constant $R$ (J\,kg$^{-1}$K$^{-1}$)is defined as:
\begin{equation}
R = \frac{\mathcal{R}}{M}
\label{eq:R_to_univR}
\end{equation}
and substituting \ref{eq:R_to_univR} into \ref{eq:univGasLaw}
\begin{equation}
p = \rho R T
\label{eq:idealGasLaw}
\end{equation}

\section*{Partial pressures}

Using Dalton's law of partial pressures, the moist air is consisted of partial pressure of dry air $p_a$ (Pa) and water vapor $p_v$ (Pa):
\begin{equation}
p = \sum_i p_i  = p_v + p_a
\label{eq:dalton}
\end{equation}
and we assume $p_o \ll p_v+p_a$ and $p_c\ll p_v + p_a$, the partial pressure of oxygen O$_2$ and CO$_2$ is negligible. We assuming that all the species are in thermal equilibrium ($T = T_v = T_a$) (K) and so:
\begin{equation}
p = \left(\rho_a R_a + \rho_v R_v \right) T
\end{equation}

The molar mass of dry air and water vapor are $M_a = \num{18.0149}$ g\,mol$^{-1}$ and $M_a=\num{28.964}$ g\,mol$^{-1}$, and $R_a = \num{287.055}$ J\,kg$^{-1}$K$^{-1}$ and $R_v = \num{461.5}$ J\,kg$^{-1}$K$^{-1}$.

Vapor pressure $p_v$ is related to relative humidity $\phi$ (-) as:
\begin{equation}
\frac{p_v}{p_{\textit{vsat}}} = \phi
\end{equation}
where $p_{\textit{vsat}}$ is the saturation vapor pressure. The relative humidity is also written as $\textrm{RH}=\phi\times 100$. The saturation vapor pressure, $p_{\textit{vsat}}$ (Pa), can be determined directly from temperature:
\begin{equation}
p_{vsat} = 610.78 \exp \left\{\frac{17.269\ (T - 273.15)}{T - 35.85}\right\}
\end{equation}

\section*{Mass Fractions}

The mass of moist gas mixture is :
\begin{equation}
m = m_v + m_a
\end{equation}
where the species of moist gas mixture is water vapor $m_v$ (kg) and dry air $m_a$ (kg).

The mass concentration $x_i$ (kg$_i$\,kg$^{-1}$) of a given species $i$ is defined as:
\begin{equation}
x_i \equiv \frac{m_i}{m}
\end{equation}
where $m_i$ (kg$_i$) is the mass of species $i$, and $m$ (kg) is the mass of the gas mixture. The mass fractions are related as:
\begin{equation}
\sum_i x_i = x_v + x_a = 1
\label{eq:totalconc}
\end{equation}

The absolute humidity or mixing ratio or humidity ratio $w$ (kg\,kg$^{-1}$) is defined as:
\begin{equation}
w \equiv \frac{m_v}{m_a} = \frac{x_v}{x_a}
\label{eq:def_w}
\end{equation}
where it the mass ratio of water vapor to dry air. Substituting, \ref{eq:totalconc} into \ref{eq:def_w} we get
\begin{equation}
w = \frac{x_v}{1 - x_v}
\label{eq:w_xv}
\end{equation}

We can rewrite \ref{eq:w_xv} to determine $x_v$ from $w$:
\begin{equation}
x_v = \frac{w}{1 + w}
\label{eq:xv_w}
\end{equation}

We can determine $w$ from $p_v$:
\begin{equation}
w = \frac{p_v/R_v T}{p_a/ R_aT} = \frac{p_v}{p_a}\frac{R_a}{R_v}
\label{eq:w_to_P}
\end{equation}
and substituting \ref{eq:dalton} into \ref{eq:w_to_P}, we get:
\begin{equation}
w = \frac{p_v}{p-p_v}\frac{R_a}{R_v}
\label{eq:pv_to_w}
\end{equation}
and rewriting \ref{eq:pv_to_w} for $p_v$, we get:
\begin{equation}
p_v = \frac{p w}{R_a/R_v+w}
\label{eq:w_to_pv}
\end{equation}

\section*{Enthalpy}

The specific enthalpy $h$ (J\,kg$^{-1}$) is defined as:

\begin{equation}
h = e + \frac{p}{\rho}
\end{equation}

where $e$ is the specific internal energy. For perfect gas, we have:
\begin{align}
\mathrm{d}e &= c_v \,\mathrm{d}T\\
\mathrm{d}h &= c_p \,\mathrm{d}T
\end{align}
where $c_v$ and $c_p$ are the specific heats at constant volume and pressure, respectively.
\begin{assumption}
	We assume that the moist air is calorically perfect gas, i.e. $c_v=\textit{constant}$, $c_p=\textit{constant}$
\end{assumption}
The specific enthalpy is therefore:
\begin{equation}
h = c_p \int^T_{T_{\textit{ref}}} \mathrm{d}T = c_p \left(T - T_{\textit{ref}}\right)
\end{equation}
where we take $T_{\textit{ref}}$ at ($0$ \si{\degreeCelsius}).
\begin{assumption}
	We assume thermal equilibrium between all the mixtures, i.e., $T = T_a = T_v$ (K) \citep{Defraeye2011}.
\end{assumption}


\begin{assumption}
	We assume binary gas mixture of dry air and water vapor and assume CO$_2$ does not play a role in conservation of energy.
\end{assumption}

Therefore, the specific enthalpy of moist air is 
\begin{equation}
h = \sum_i x_i h_i = x_a h_a + x_v h_v
\end{equation}
where $x_a$ and $x_v$ (kg\,kg$^{-1}$) are dry air and vapor mass concentration, and $h_a$ and $h_v$ (J\,kg$^{-1}$) are:
\begin{equation}
h_a = c_{pa} \left(T - T_{\textit{ref}}\right)
\end{equation}
and 
\begin{equation}
h_v = c_{pv} \left(  T - T_{\textit{ref}}\right) + L_v
\end{equation}
where $L_v$ (J\,kg$^{-1}$) is the latent heat of vaporization (or more accurately referenced at a reference temperature). The reference temperature $T_{\textit{ref}}$ (K) is usually taken to be $0^{\circ}$C, i.e., $T_{\textit{ref}} = 273.15$ K and the latent heat of vaporization taken at the same temperature $L_v = \num{2.5e6}$ J\,kg$^{-1}$ at $0^{\circ}$C. Thus the specific heat of moist air $h$ (J\,kg${-1}$) becomes:
\begin{equation}
h = \left(x_a c_{pa} + x_v c_{pv} \right) \left(T - T_{\textit{ref}}\right)  + x_v L_v
\label{eq:enthalpymoistair}
\end{equation}

Typically, for atmospheric flows with relatively low water vapor concentration, we have $x_v c_{pv} \ll x_a c_{pa}$ and so: 

\begin{equation}
h \approx x_a c_{pa} \left(T - T_{\textit{ref}}\right)  + x_v L_v
\end{equation}

